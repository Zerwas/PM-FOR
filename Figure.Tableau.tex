\begin{tikzpicture}[grow=up,level distance=0.5cm,
execute at begin node=$, execute at end node=$,
every node/.style={opacity=1},
every child/.style={edge from parent/.style={opacity=0}}]
\def\dist{0.6cm}
\node(e) {\Gamma_C\cup\conGM\PModelsf \false} [sibling distance=2*\dist+2.8cm] 
	child {node(1) {\Gamma_C\cup\conGM\PModelsf \Gamma_D}}
	child {node(2) {\Gamma_C\cup\conGM\cup \Gamma_D\PModelsf\false}
		child {node(21) {\text{Induction Hypothesis}}}};


%tikz did not want to do this in the loop no idea why
%\coordinate(c1222) at (1222);
%\getwidthofnode{\sright}{12221}
%\draw (c1222)+(-\sright/2,+0.25) -- +(\sright/2,+0.25);

%set of all positions in the tree
\def\nodes{,1,2,11,12,21}
%the yshift does not work on nodes so we create a coordinate for every node
\foreach \x in \nodes{
\ifnodedefined{\x}{
	\coordinate(c\x) at (\x);
}{}
}

%draw lines for each node with 2 children (add parents with 2 children)
\foreach \x in \nodes{
\ifnodedefined{\x1}{
	%calculate with of line
	\getwidthofnode{\sright}{\x1}
	%if there is a right child draw wide line
	\ifnodedefined{\x2}{
		\getwidthofnode{\sleft}{\x2}
		%draw line
		\draw[shorten <=-\sright/2, shorten >=-\sleft/2] ([yshift=-0.25cm]c\x1) -- ([yshift=-0.25cm]c\x2);
	} %else only one child
	{
		\getwidthofnode{\sleft}{\x}
		%get widht corresponding to the widht of the bigger node
		\pgfmathsetlength{\sright}{max(\sright,\sleft)}
		%draw line
		\draw (c\x)+(-\sright/2,+0.25) -- +(\sright/2,+0.25);
	}
}{}
}
\end{tikzpicture}