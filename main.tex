\documentclass[]{scrartcl}

\usepackage[utf8]{inputenc}
\usepackage[english]{babel}
\usepackage{amsmath}
\usepackage{amsthm}
\usepackage{thmtools}
\usepackage{amssymb}
\usepackage{float}
\usepackage{tikz}
\usepackage{calc}
\usepackage[normalem]{ulem}
\usepackage[framemethod=TikZ]{mdframed}
\usepackage{enumitem}

\usetikzlibrary{arrows}

\newtheorem{theorem}{Theorem}
\newtheorem{lemma}[theorem]{Lemma}
\newtheorem{claim}[theorem]{Claim}
\theoremstyle{definition}
\newtheorem{definition}[theorem]{Definition}

\newcommand{\define}[1]{\uline{#1}}

%System P
\newcommand{\PCons}{\textbf{CONS}}
\newcommand{\Pformulas}{\ensuremath{\mathcal{L}_{(\VarP,\RelP)}}}
\newcommand{\SysP}{\textbf{P}}
\newcommand{\false}{\textbf{false}}
\newcommand{\falses}{\textbf{f}} %false short
\newcommand{\VarP}{\ensuremath{\mathcal{V}_P}}
\newcommand{\RelP}{\ensuremath{\mathcal{P}_P}}
\newcommand{\vdashf}{\vdash_\text{f}}
%Lambda2
\newcommand{\lambdaInhab}{\textbf{INHAB}}
\newcommand{\lambdaTypes}{\ensuremath{\text{T}_{\lambda2}}}
\newcommand{\lambdaTerms}{\ensuremath{\Lambda_{\lambdaTypes}}}
\newcommand{\lambdaTypVar}{\ensuremath{\mathcal{V}_T}}
\newcommand{\lambdaValVar}{\ensuremath{\mathcal{V}_V}}
%first-order logic
\newcommand{\rank}{\emph{rk}}
\newcommand{\folmodels}{\ensuremath{\vdash}}
\newcommand{\V}{\text{V}}
\newcommand{\FV}{\text{FV}}
%two-counter automaton
\newcommand{\autHalt}{\textbf{HALT}}
\newcommand{\autStates}{\ensuremath{\mathcal{Q}}}
\newcommand{\autRules}{\ensuremath{R}}
%construction
\newcommand{\conGM}{\ensuremath{\Gamma_{\overline{M}}}}

\makeatletter
\newcommand\getwidthofnode[2]{%
    \pgfextractx{#1}{\pgfpointanchor{#2}{east}}%
    \pgfextractx{\pgf@xa}{\pgfpointanchor{#2}{west}}% \pgf@xa is a length defined by PGF for temporary storage. No need to create a new temporary length.
    \addtolength{#1}{-\pgf@xa}%
}

\long\def\ifnodedefined#1#2#3{%
    \@ifundefined{pgf@sh@ns@#1}{#3}{#2}%
}
\makeatother


%define lengths for tikz
\newlength{\sleft}
\newlength{\sright}

\begin{document}
	\binoppenalty=10000
	\relpenalty=10000
	\tableofcontents
	\begin{sloppypar}
	\newpage
	\begin{abstract}
asdfaisdkok
\end{abstract}

\section{Introduction}\label{sec.1}


	\section{Basic Definitions}
We will denote the set $\{1,\dots,n\}$ by $\left[n\right]$.
\begin{definition}
A \uline{ranked set} is a tuple $(\Sigma,\rank)$, where $\Sigma$ is a countable set and $\rank:\Sigma\to\mathbb{N}$ is a function that maps every symbol from $\Sigma$ to a natural number (its rank).
\end{definition}
If the function \rank is understood we will just write $\Sigma$ instead of $(\Sigma,\rank)$. The set of all elements with a certain rank $k$ in $\Sigma$, denoted by $\Sigma^{(k)}$, is defined by $\Sigma^{(k)}:=rk^{-1}(k)$. In the following we will write $\Sigma=\{P^{(0)},Q^{(3)}\}$ to say that $\Sigma=\{P,Q\}$, $\rank(P)=0$, and $\rank(Q)=3$.
\subsection{First-order logic}
Let $\mathcal{V}=\{x_0,x_1,\dots\}$ be a countable set (of variables), $\mathcal{F}=\{\}$ a ranked set (of function symbols), and $\mathcal{P}=\{\}$ a ranked set (of predicate symbols). The first-order formulas over $(\mathcal{V},\mathcal{F},\mathcal{P})$, are defined as follows:
	\section{P-System}
$\emph{FV}(\Gamma)=\bigcup\{\emph{FV}(A)\mid A\in \Gamma\}$\\
Deduction Rules
\begin{mdframed}
\begingroup
\addtolength{\jot}{0.3cm}
\begin{align*}
&(\text{Axiom}) &&\Gamma,A\vdash A\\
&(\rightarrow\text{-Introduction}) &&\frac{\Gamma,A\vdash B}{\Gamma\vdash A\to B}\\
&(\rightarrow\text{-Elimination}) &&\frac{\Gamma\vdash A\to B \hspace{0.4cm}\Gamma\vdash A}{\Gamma\vdash B}\\
&(\forall\text{-Introduction}) &&\frac{\Gamma\vdash B}{\Gamma\vdash \forall\alpha B} &&\alpha\notin\emph{FV}(\Gamma) \\
&(\forall\text{-Elimination}) &&\frac{\Gamma\vdash \forall\alpha B }{\Gamma\vdash B\left[ \alpha:=\beta\right] } %TODO dot after \alpha?
\end{align*}
\endgroup
\end{mdframed}

	\end{sloppypar}
\end{document}
