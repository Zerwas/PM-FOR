\section{System P}
\subsection{Definitions}
%TODO mention the convention that a free and \alpha not or change the convention :D
In the following let $\VarP=\{\alpha,a,\beta,b,\dots\}$ be a countably infinite set (of variables). 
Let $\RelP=\{P,Q,\dots\}$ be a set (of predicate symbols) and $\mathcal{P}$ a ranked set such that $\mathcal{P}^{(0)}=\{\false\}$, $\mathcal{P}^{(2)}=\RelP$, and $\mathcal{P}^{(k)}=\emptyset$ for all $k\in\mathbb{N}\setminus\{0,2\}$.
A first-order logic formula $\varphi$ over $(\VarP,\emptyset,\mathcal{P})$ is an 
%TODO \RelP not countybly infinite since Q is finite
%todo later new \RelP so only bind it locally
\begin{description} %TODO \overrightarrow{\alpha}
	\item[atomic formula] if $\varphi=\false$ or $\varphi=P(a,b)$ for some $P\in\RelP$ and $a,b\in\VarP$.
	\item[universal formula] if $\varphi=\forall\overrightarrow{\alpha}(A_1\to A_2 \to\dots\to A_n)$ where $A_i$ is an atomic formula for $i\in\left[n\right]$, $A_i\neq\false$ for $i\in\left[n-1\right]$ and for each $\alpha\in\FV(\varphi)\cap\FV(A_n)$ there exists an $i\in\left[n-1\right]$ such that $\alpha\in\FV(A_i)$.
	\item[existential formula] if there exits $n\ge0$,  atomic formulas $A_i\neq\false$ for $i\in\left[n\right]$ such that $\varphi=\forall\overrightarrow{\alpha}(A_1\to A_2 \to\dots\to A_{n-1}\to\forall\beta(A_n\to\false)\to\false)$.
\end{description}
The set of formulas of System \SysP{} (= set of \SysP-formulas) over $(\VarP,\RelP)$ is the set of all first-order formulas over $(\VarP,\emptyset,\mathcal{P})$ that are either an atomic, universal or existential formula.\\ 
Deduction Rules
\begin{mdframed}
	\begingroup%TODO maybe not A and B for terms?
	\addtolength{\jot}{0.3cm}
	\begin{align*}
		&(\text{Axiom}) &&\Gamma,A\vdash A\vphantom{\frac{\Gamma}{\Gamma}}\\
		&(\rightarrow\text{-Introduction}) &&\frac{\Gamma,A\vdash B}{\Gamma\vdash A\to B}\\
		&(\rightarrow\text{-Elimination}) &&\frac{\Gamma\vdash A\to B \hspace{0.4cm}\Gamma\vdash A}{\Gamma\vdash B}\\
		  & (\forall\text{-Introduction}) &   & \frac{\Gamma\vdash B}{\Gamma\vdash \forall\alpha B} &   & \alpha\notin\FV(\Gamma) \\
		&(\forall\text{-Elimination}) &&\frac{\Gamma\vdash \forall\alpha B }{\Gamma\vdash B\left[ \alpha:=b\right] }
		&& b\in\VarP %TODO dot after \alpha?, b\in\VarP necessary?
	\end{align*}
	\endgroup
\end{mdframed}
An Interpretation $I$ of a P formula is a tuple $I=(\Delta,\cdot^I)$ where $\Delta$ is a set (called domain), $P^I\subseteq\Delta^k$ and $\alpha^I\in\Delta$.\dots\\
If we interpret \false{} with the logical constant false ($\bot$) (denoted by $\vdash_f$) we can add a new deduction rule.
\begin{mdframed}
	\begin{align*}
		  & (\exists\text{-Introduction}) &   & \frac 
		{\Gamma,A\left[\alpha:=a\right]\vdashf B}
		{\Gamma,\forall\alpha(A\to\false)\to\false\vdashf B} && a\notin\emph{FV}(\Gamma,A,B)
	\end{align*}
\end{mdframed}
\begin{proof}
	Let $I=(\Delta,\cdot^I,\omega)$ be a model of $\Gamma,\forall\alpha(A\to\false)\to\false$ with $\false^I=\bot$ and $a\in\VarP$ a variable such that $a\notin\emph{FV}(\Gamma,A,B)$.
	\begin{align*}
		I\models\Gamma,\forall\alpha(A\to\false)\to\false & \Rightarrow I\models\forall\alpha(A\to\false)\to\false                                                          \\
        & \Rightarrow (\forall\alpha(A\to\false))^I\to\false^I                                                            \\
        & \Rightarrow (\forall\alpha(A\to\false))^I\to\bot                                                                \\
        & \Rightarrow \neg(\forall\alpha(A\to\false))^I                                                                   \\
        & \Rightarrow \neg(\forall d\in\Delta:(A\to\false)^{I\left[\alpha\mapsto d\right]})                               \\
        & \Rightarrow \exists d\in\Delta:\neg(A^{I\left[\alpha\mapsto d\right]}\to\false^{I\left[\alpha\mapsto d\right]}) \\
        & \Rightarrow \exists d\in\Delta:\neg(A^{I\left[\alpha\mapsto d\right]}\to\bot)                                   \\
        & \Rightarrow \exists d\in\Delta:\neg(\neg A^{I\left[\alpha\mapsto d\right]})                                     \\
        & \Rightarrow \exists d\in\Delta:A^{I\left[\alpha\mapsto d\right]}
	\end{align*}
	Together with $a\notin\emph{FV}(\Gamma,A)$, it follows that $I\left[a\mapsto d\right]$ is a model of $\Gamma,A\left[\alpha:=a\right]$. Which implies $I\left[a\mapsto d\right]\models B$.	Since $a$ is not free in $B$ we conclude that $I$ is also a model of $B$.
\end{proof}
\begin{definition}
	The problem to decide whether a given set of \SysP-formulas is consistent, denoted by \PCons, is defined as follows.
	Given a set of \SysP-formulas $\Gamma$. 
	\begin{center}
		Does $\Gamma\vdash\false$ not hold.
	\end{center}
\end{definition}
\subsection{\PCons{} is undecidable}
We will show that $\autHalt\leq\PCons$ then the undecidability of \PCons{} directly follows from the undecidability of \autHalt. For a given two-counter automaton $M$ we will effectively construct a set of \SysP-formulas $\Gamma_M$ such that
\begin{align*}
 & \text{$M$ terminates on input $(0,0)$} &   & \text{iff} & \text{$\Gamma_M\vdash\false$ holds in system P.} 
\end{align*}
Let $M=(\autStates,Q_0,Q_f,\autRules)$ be a two-counter automaton, w.l.o.g. $S,P,R_1,R_2,E,D\notin\autStates$. In the following we will consider \SysP-formulas over $(\VarP,\RelP)$, where $\RelP=\autStates\uplus\{S,P,R_1,R_2,E,D\}$. We will abbreviate $P(a,a)$ to $P(a)$, note that this way we can use binary predicate symbols as unary ones.

Intuitively $Q(a)$ stands for ``$a$ is in state $Q$", $R_i(a,m)$ stands for ``in $a$ the value of register i is $m$" for $i\in\{1,2\}$, $S(a,b)$ states that ``$b$ is a successor of $a$", $P(a,b)$ states that ``$b$ is a predecessor of $a$", $E(a)$ marks ``$a$ as the end of chain", and $D(a)$ states that ``$a$ is not the end of a chain".

For a configuration $C=\langle Q,m,n\rangle$ of $M$ we define a set of \SysP-formulas $\Gamma_C$. It contains the following formulas:
\begin{itemize}
	\item $Q(a)$
	\item $R_1(a,a_0),P(a_{i-1},a_i)~\text{for}~i\in\{1,\dots,m\}$
	\item $R_2(a,b_0),P(b_{i-1},b_i)~\text{for}~i\in\{1,\dots,n\}$
	\item $D(a),D(a_i),D(b_j)~\text{for}~i\in\{0,\dots,m-1\}~\text{and}~j\in\{0,\dots,n-1\}$
	\item $E(a_m),E(b_n)$
\end{itemize}
Next we need sets of \SysP-formulas for all possible transitions.
For every $Q\in\autStates\setminus\{Q_f\}$ and $r\in\mathcal{R}_\autStates$ we define $\Gamma_{Q,r}$.
If $r=+(1,Q')$ for some $Q'\in\autStates$ then $\Gamma_{Q,+(1,Q')}$ contains the following formulas:
\begin{itemize}
	\item $\forall\alpha\beta(Q(\alpha)\to S(\alpha,\beta)\to Q'(\beta))$ \\change of state
	\item $\forall\alpha\beta\gamma\delta(Q(\alpha)\to S(\alpha,\beta)\to R_1(\alpha,\gamma)\to R_1(\beta,\delta)\to P(\delta,\gamma))$\\increment register 1
	\item $\forall\alpha\beta\delta(Q(\alpha)\to S(\alpha,\beta)\to R_1(\beta,\delta)\to D(\delta))$ \\prevent zero in register 1
	\item $\forall\alpha\beta\gamma(Q(\alpha)\to S(\alpha,\beta)\to R_2(\alpha,\gamma)\to R_2(\beta,\gamma))$ \\do not change the value register 2
\end{itemize}

If $r=-(1,Q_1,Q_2)$ for some $Q_1,Q_2\in\autStates$ then $\Gamma_{Q,-(1,Q_1,Q_2)}$ contains the following formulas:
\begin{itemize}
	\item $\forall\alpha\beta\gamma(Q(\alpha)\to S(\alpha,\beta)\to R_1(\alpha,\gamma)\to E(\gamma)\to Q_2(\beta))$\\jump to $Q_2$ if register 1 is zero
	\item $\forall\alpha\beta\gamma(Q(\alpha)\to S(\alpha,\beta)\to R_1(\alpha,\gamma)\to E(\gamma)\to R_1(\beta,\gamma))$\\if register 1 is zero it stays zero
	\item $\forall\alpha\beta\gamma(Q(\alpha)\to S(\alpha,\beta)\to R_1(\alpha,\gamma)\to D(\gamma)\to Q_1(\beta))$\\change state to $Q_1$ if register 1 is greater zero
	\item $\forall\alpha\beta\gamma\delta(Q(\alpha)\to S(\alpha,\beta)\to R_1(\alpha,\gamma)\to D(\gamma) \to P(\gamma,\delta)\to R_1(\beta,\delta))$\\decrement register 1
	\item $\forall\alpha\beta\gamma(Q(\alpha)\to S(\alpha,\beta)\to R_2(\alpha,\gamma)\to R_2(\beta,\gamma))$\\do not change register 2 in both cases
\end{itemize}

For $r=+(2,Q')$ for some $Q'\in\autStates$ or $r=-(2,Q_1,Q_2)$ for some $Q_1,Q_2\in\autStates$ the sets $\Gamma_{Q,r}$ are defined analogously.

We also need a set $\Gamma_1$ to ensure that our representation works correctly. The following formula are in $\Gamma_1$:
\begin{itemize}
	\item $\forall\alpha\beta(S(\alpha,\beta)\to D(\beta))$\\no successor is the end of a chain
	%\item $\forall\alpha(D(\alpha)\to\forall\beta(P(\alpha,\beta)\to\false)\to\false)$\\every %element that is not the end of a chain has a predecessor
	%TODO necessary?
	\item $\forall\alpha(D(\alpha)\to\forall\beta(R_1(\alpha,\beta)\to\false)\to\false)$\\every element that represents a configuration has a value for register 1
	\item $\forall\alpha(D(\alpha)\to\forall\beta(R_2(\alpha,\beta)\to\false)\to\false)$\\every element that represents a configuration has a value for register 2
	\item $\forall\alpha(\forall\beta(S(\alpha,\beta)\to\false)\to\false)$\\every element has a successor
\end{itemize}
We define $\conGM$ as $\bigcup_{Q\in\autStates\setminus\{Q_f\}}\Gamma_{Q,R(Q)}\cup\{\forall\alpha(Q_f(\alpha)\to\false)\}\cup\Gamma_1$.
Finally we can define $\Gamma_M$ as $\Gamma_{C_1}\cup\conGM$, where $C_1=\langle Q_0,0,0\rangle$ is the initial configuration.
\begin{claim}\label{cla.17}
	\begin{align*}
		  & \text{$\Gamma_M\vdash\false$ holds in system P} &   & \implies & \text{$M$ terminates on input $(0,0)$} 
	\end{align*}
\end{claim}
\begin{proof}
	Assume $M$ does not terminate then there is an infinite chain $C_1\Rightarrow_M C_2\Rightarrow_M C_3\Rightarrow_M\dots$ ($C_i=\langle Q_i,m_i,n_i\rangle$ for $i\in\mathbb{N}^+$). Now we construct a model of $\Gamma_M$ which interprets \false{} with $\bot$ this contradicts $\Gamma_M\vdash\false$.
	
	To illustrate the idea we will use a graphical notation for an interpretation $I$.
	By
	\tikz[baseline=-3pt]{
		\node (1) {$d_1$};
		\node[right of=1] (2) {$d_2$};
		\path[->,>=latex'] (1) edge node[above,scale=0.8] {R} (2);}
	we say that $(d_1,d_2)\in R^I$. And we use
	\tikz[baseline=5pt]{
		\node (1) {$d$};
		\node[node distance = 0.6cm,above of=1,scale=0.8] {P};}
	to say that $(d,d)\in P^I$ for predicate symbols that are used as unary predicate symbols. 
	As domain for our interpretation we will use the natural numbers. Every number will have two tasks: firstly it will represent itself as a possible value for register 1 or 2 and secondly every number $i$ greater than zero will also represent the $i$\textsuperscript{th} configuration of our infinite computation.
	Now the idea for our model of $\Gamma_M$ looks like this:
	%TODO is it really usefull that 0 does not represent a configuration? it dous not look like it
	
	\begin{figure}[H]
		\centering
		\begin{tikzpicture}[execute at begin node=$, execute at end node=$,
label/.style={scale=0.8},
every path/.style={->,>=latex'},node distance = 1.4cm]
\node (0) {0};
\node[right of=0] (1) {1};
\node[right of=1] (2) {\dots};
\node[right of=2] (3) {m_i};
\node[right of=3] (4) {\dots};
\node[right of=4] (5) {n_i};
\node[right of=5] (6) {\dots};
\node[right of=6] (7) {i};
\node[right of=7] (8) {\dots};


\node[node distance = 0.6cm,above of=1,label]     {Q_1};
\node[node distance = 0.6cm,above of=3,label]     {Q_{m_i}};
\node[node distance = 0.6cm,above of=5,label]     {Q_{n_i}};
\node[node distance = 0.6cm,above of=7,label]     {Q_i};


\foreach \x in {0,...,7}{
\pgfmathtruncatemacro{\xn}{\x+1}
\path (\xn) edge[bend left=10] node[below,label] {P} (\x);
\path (\x) edge[bend left=10] node[above,label] {S} (\xn);
}

\path (1) edge[bend right=60] node[above,label] {R_1} (0);
\path (1) edge[bend left=60] node[below,label] {R_2} (0);

\path (7) edge[bend right=50] node[above,label] {R_1} (3);
\path (7) edge[bend left=60] node[below,label] {R_2} (5);


\end{tikzpicture}
	\end{figure}
	We have $0\in E^I$ and all other numbers are in $D^I$.
	
	Here is the more formal definition of our model $I=(\mathbb{N},\cdot^I,\omega)$.
	\begin{align*}
		  P^I&=\{(i+1,i)\mid i\in\mathbb{N}\}              & R_1^I&=\{(i,m_i)\mid i\in\mathbb{N}\} & R_2^I&=\{(i,n_i)\mid i\in\mathbb{N}\} \\
		  %Q^I&=\{ i\in\mathbb{N}\setminus\{0\}\mid Q=Q_i\} &  D^I&=\mathbb{N}\setminus\{0\}         & E^I&=\{0\}                            \\
		  Q^I&=\{(i,i)\mid i\in\mathbb{N}^+, Q=Q_i\} &  D^I&=\{(i,i)\mid i\in\mathbb{N}^+\}         & E^I&=\{(0,0)\}                            \\
		  S^I&=\{(i,i+1)\mid i\in\mathbb{N}\} & \false^I&=\bot
	\end{align*}
	\begin{align*}
		  & a^I=1 &   & a_0^I=0 &   & b_0^I=0 
	\end{align*}
	Since there are no free variables in $\Gamma_M$ we can just set $\omega(x)=0$ for every $x\in\VarP$. It is easy to see that $I$ is indeed a model of $\Gamma_M$.
\end{proof}
\begin{claim}\label{cla.18}
	Let $C$ be a configuration of $M$. If a final configuration (i.e. a configuration $\langle Q_f,m,n\rangle$ for some $n,m\in\mathbb{N}$) is reachable from $C$ then $\Gamma_C\cup\conGM\vdash\false$.
\end{claim}
\begin{proof} By induction on the length $n$ of the computation.
	For the tableau proofs we will abbreviate \false{} by \falses.
	
	Induction Base: $n=0$\\
	Hence, a final configuration is reachable in 0 steps, $C$ must be this final configuration. So $C=\langle Q_f,m,n\rangle$ for some $n,m\in\mathbb{N}$. Since $Q_f(a)$ is in $\Gamma_C$ for some $a\in\VarP$ and $\forall\alpha(Q_f(\alpha)\to\false)$ is in $\conGM$ we can easily deduce false.
	\begin{figure}[H]
		\centering
		\begin{tikzpicture}[grow=up,
execute at begin node=$, execute at end node=$,
every node/.style={opacity=1},
every child/.style={edge from parent/.style={opacity=0}}]
\def\dist{0.6cm}
\node(e) {\Gamma_C\cup\conGM\vdash \false} [sibling distance=2*\dist+3.8cm] 
	child {node(1) {\Gamma_C\cup\conGM\vdash Q_f(a)}}
	child {node(2) {\Gamma_C\cup\conGM\vdash Q_f(a)\to\false}
		child {node(21) {\Gamma_C\cup\conGM\vdash\forall\alpha(Q_f(\alpha)\to\false)}}};


%tikz did not want to do this in the loop no idea why
%\coordinate(c1222) at (1222);
%\getwidthofnode{\sright}{12221}
%\draw (c1222)+(-\sright/2,+0.25) -- +(\sright/2,+0.25);

%set of all positions in the tree
\def\nodes{,1,2,11,12,21}
\def\identifier{0}
\foreach \x in \nodes{
\ifnodedefined{\x}{
	\coordinate(c\x) at (\x);
	\coordinate(\identifier.\x) at (0,0);
}{}
}

%draw lines for each node with 2 children (add parents with 2 children)
\foreach \x in \nodes{
\ifnodedefined{\identifier.\x1}{
	%calculate with of line
	\getwidthofnode{\sright}{\x1}
	%if there is a right child draw wide line
	\ifnodedefined{\identifier.\x2}{
		\getwidthofnode{\sleft}{\x2}
		%draw line
		\draw[shorten <=-\sright/2, shorten >=-\sleft/2] ([yshift=-0.25cm]c\x1) -- ([yshift=-0.25cm]c\x2);
	} %else only one child
	{
		\getwidthofnode{\sleft}{\x}
		%get widht corresponding to the widht of the bigger node
		\pgfmathsetlength{\sright}{max(\sright,\sleft)}
		%draw line
		\draw (c\x)+(-\sright/2,+0.25) -- +(\sright/2,+0.25);
	}
}{}
}
\end{tikzpicture}
	\end{figure}
	 
	Induction Step: $n\to n+1$\\
	$C\Rightarrow_M^r D$\\
	We need to make a case distinction on the rule $r$.\\
	\underline{Case $r=+(Q,1,Q')$}\\
	
	Basic idea:\\ %TODO tiksify
	\[
		\cfrac
		{\cfrac{IH}{\Gamma_C\cup\conGM\cup\Gamma_D\vdash\falses}\hspace{0.6cm}
			\cfrac{}{\Gamma_C\cup\conGM\vdash\Gamma_D}}
		{\Gamma_C\cup\conGM\vdash\falses}
	\]
	Since $I\models\false$ holds trivially if $I$ interprets \false{} with $\top$ we only need to consider models (note that there are none if $M$ terminates which is exactly what we want to proof) of $\Gamma_C\cup\conGM$ that interpret \false{} with $\bot$ (so we can use our new deduction rule).\\
	We will just drop $\Gamma_C\cup\conGM$ and only write new formulas on the left side.\\
	We first introduce the new variables needed for $\Gamma_D$ (let $b,d\in \VarP\setminus\text{FV}(\Gamma_C\cup\conGM)$). Intuitively $b$ will represent the successor state and $d$ will be the anchor for register one.
	
	\begin{figure}[H]
		\centering
		\begin{tikzpicture}[grow=up,level distance=0.5cm,
execute at begin node=$, execute at end node=$,
every node/.style={opacity=1},
every child/.style={edge from parent/.style={opacity=0}}]
\node(e) {\PModelsf \falses} [sibling distance=8cm] 
	child [sibling distance=4cm] {node(1) {\PModelsf\forall\beta(S(a,\beta)\to\falses)\to\falses} 
		child {node(11) {\PModelsf\forall\alpha(\forall\beta(S(\alpha,\beta)\to\falses)\to\falses)}}}
	child {node(2) {\PModelsf(\forall\beta(S(a,\beta)\to\falses)\to\falses)\to\falses}
		child {node(21) {\forall\beta(S(a,\beta)\to\falses)\to\falses\PModelsf\falses}
			child {node(211) {S(a,b)\PModelsf\falses}
				child {node[label={[yshift=-0.3cm]above:\vdots}](2111) {}}}}};

%set of all positions in the tree
\def\nodes{,1,2,11,21,211,2111}
%the yshift does not work on nodes so we create a coordinate for every node
\foreach \x in \nodes{
\ifnodedefined{\x}{
	\coordinate(c\x) at (\x);
	\coordinate(1.\x) at (0,0); %we need this because defined nodes are not fergotten between tikz pictures
}{}
}

%draw lines for each node with 2 children (add parents with 2 children)
\foreach \x in \nodes{
\ifnodedefined{1.\x1}{
	%calculate with of line
	\getwidthofnode{\sright}{\x1}
	%if there is a right child draw wide line
	\ifnodedefined{1.\x2}{
		\getwidthofnode{\sleft}{\x2}
		%draw line
		\draw[shorten <=-\sright/2, shorten >=-\sleft/2] ([yshift=-0.25cm]c\x1) -- ([yshift=-0.25cm]c\x2);
	} %else only one child
	{
		\getwidthofnode{\sleft}{\x}
		%get widht corresponding to the widht of the bigger node
		\pgfmathsetlength{\sright}{max(\sright,\sleft)}
		%draw line
		\draw (c\x)+(-\sright/2,+0.25) -- +(\sright/2,+0.25);
	}
}{}
}

\end{tikzpicture}
	\end{figure}
	The formula $R_1(b,d)$ can be acquired in a similar way. Again we will just drop $S(a,b)$ and $D(b)$ on the left side for comprehensibility.
	
	\begin{figure}[H]
		\centering
		\begin{tikzpicture}[grow=up,level distance=0.5cm,
execute at begin node=$, execute at end node=$,
every node/.style={opacity=1},
every child/.style={edge from parent/.style={opacity=0}}]
\node(e) {\PModelsf\falses} [sibling distance=6.5cm]
	child {node(1) {\PModelsf\forall\beta(R_1(b,\beta)\to\falses)\to\falses} [sibling distance=4cm]
		child {node(11) {\PModelsf\forall\alpha(\forall\beta(R_1(\alpha,\beta)\to\falses)\to\falses)}}}
	child {node(2) {\PModelsf(\forall\beta(R_1(b,\beta)\to\falses)\to\falses)\to\falses}
		child {node(21) {\forall\beta(R_1(b,\beta)\to\falses)\to\falses\PModelsf\falses}
			child {node(211) {R_1(b,d)\PModelsf\falses}
				child {node[label={[yshift=-0.3cm]above:\vdots}](2111) {}}}}};

%set of all positions in the tree
\def\nodes{,1,2,11,21,211,2111}
%the yshift does not work on nodes so we create a coordinate for every node
\foreach \x in \nodes{
\ifnodedefined{\x}{
	\coordinate(c\x) at (\x);
	\coordinate(2.\x) at (0,0);
}{}
}

%draw lines for each node with 2 children (add parents with 2 children)
\foreach \x in \nodes{
\ifnodedefined{2.\x1}{
	%calculate with of line
	\getwidthofnode{\sright}{\x1}
	%if there is a right child draw wide line
	\ifnodedefined{2.\x2}{
		\getwidthofnode{\sleft}{\x2}
		%draw line
		\draw[shorten <=-\sright/2, shorten >=-\sleft/2] ([yshift=-0.25cm]c\x1) -- ([yshift=-0.25cm]c\x2);
	} %else only one child
	{
		\getwidthofnode{\sleft}{\x}
		%get widht corresponding to the widht of the bigger node
		\pgfmathsetlength{\sright}{max(\sright,\sleft)}
		%draw line
		\draw (c\x)+(-\sright/2,+0.25) -- +(\sright/2,+0.25);
	}
}{}
}

\end{tikzpicture}
	\end{figure}
	Now we have all the new free variables we need and we continue by ensuring that these variables fulfill all the formulas in $\Gamma_D$.
	
	\begin{figure}[H]
		\centering
		\begin{tikzpicture}[grow=up,level distance=0.5cm,
execute at begin node=$, execute at end node=$,
every node/.style={opacity=1},
every child/.style={edge from parent/.style={opacity=0}}]
\node(e) {\vdashf \falses} [sibling distance=12.4cm] 
	child [sibling distance=4cm] {node(1) {\vdashf Q'(b)} 
		child {node(11) {\vdashf S(a,b)} }
		child [sibling distance=3.7cm] {node(12) {\vdashf S(a,b)\to Q'(b)} 
			child {node(121) {\vdashf Q(a)} }
			child {node(122) {\vdashf Q(a)\to S(a,b)\to Q'(b)} 
				child {node(1221) {\vdashf\forall\alpha\beta(Q(\alpha)\to S(\alpha,\beta)\to Q'(\beta))} }}}}
	child {node(2) {\vdashf Q'(b)\to\falses} 
		child {node(21) {Q'(b)\vdashf\falses} 
			child {node[label={[yshift=-0.3cm]above:\vdots}](211) {}}}};

%set of all positions in the tree
\def\nodes{,1,2,11,12,21,121,122,1221,211}
%the yshift does not work on nodes so we create a coordinate for every node
\foreach \x in \nodes{
\ifnodedefined{\x}{
	\coordinate(c\x) at (\x);
	\coordinate(3.\x) at (0,0);
}{}
}

%draw lines for each node with 2 children (add parents with 2 children)
\foreach \x in \nodes{
\ifnodedefined{3.\x1}{
	%calculate with of line
	\getwidthofnode{\sright}{\x1}
	%if there is a right child draw wide line
	\ifnodedefined{3.\x2}{
		\getwidthofnode{\sleft}{\x2}
		%draw line
		\draw[shorten <=-\sright/2, shorten >=-\sleft/2] ([yshift=-0.25cm]c\x1) -- ([yshift=-0.25cm]c\x2);
	} %else only one child
	{
		\getwidthofnode{\sleft}{\x}
		%get widht corresponding to the widht of the bigger node
		\pgfmathsetlength{\sright}{max(\sright,\sleft)}
		%draw line
		\draw (c\x)+(-\sright/2,+0.25) -- +(\sright/2,+0.25);
	}
}{}
}
\end{tikzpicture}
	\end{figure}
	Starting from $Q'(b)\vdashf\false$ we can connect $d$ and $a_0$.
	
	\begin{figure}[H]
		\centering
		\begin{tikzpicture}[grow=up,level distance=0.5cm,
execute at begin node=$, execute at end node=$,
every node/.style={opacity=1},
every child/.style={edge from parent/.style={opacity=0}}]
\def\dist{0.4cm}
\node(e) {\PModelsf \falses} [sibling distance=2*\dist+8cm] 
	child [sibling distance=\dist+3cm] {node(1) {\PModelsf P(d,a_0)} 
		child {node(11) {\PModelsf R_1(b,d)} }
		child [sibling distance=\dist+3.9cm] {node(12) {\PModelsf R_1(b,d)\to P(d,a_0)} 
			child {node(121) {\PModelsf R_1(a,a_0)} }
			child [sibling distance=\dist+4.5cm] {node(122) {\PModelsf R_1(a,a_0)\to R_1(b,d)\to P(d,a_0)} 
				child {node(1221) {\PModelsf S(a,b)} }
				child [sibling distance=\dist+5cm] {node(1222) {\PModelsf S(a,b)\to R_1(a,a_0)\to R_1(b,d)\to P(d,a_0)}
					child {node(12221) {\PModelsf Q(a)} }
					child {node(12222) {\PModelsf Q(a)\to S(a,b)\to R_1(a,a_0)\to R_1(b,d)\to P(d,a_0)}
						child {node(122221) {\PModelsf \forall\alpha\beta\gamma\delta(Q(\alpha)\to S(\alpha,\beta)\to R_1(\alpha,\gamma)\to R_1(\beta,\delta)\to P(\delta,\gamma))}}}
					child {}}
				child {}}
			child {}}
		child {}}
				%child {node(1221) {\PModelsf\forall\alpha\beta(Q(\alpha)\to S(\alpha,\beta)\to Q'(\beta))} }}}}
	child {node(2) {\PModelsf P(d,a_0)\to\falses} 
		child {node(21) {P(d,a_0)\PModelsf\falses} 
			child {node[label={[yshift=-0.3cm]above:\vdots}](211) {}}}};

%set of all positions in the tree
\def\nodes{,1,2,11,12,21,121,122,1221,1222,12221,12222,122221,211}
%the yshift does not work on nodes so we create a coordinate for every node
\foreach \x in \nodes{
\ifnodedefined{\x}{
	\coordinate(c\x) at (\x);
	\coordinate(4.\x) at (0,0);
}{}
}

%draw lines for each node with 2 children (add parents with 2 children)
\foreach \x in \nodes{
\ifnodedefined{4.\x1}{
	%calculate with of line
	\getwidthofnode{\sright}{\x1}
	%if there is a right child draw wide line
	\ifnodedefined{4.\x2}{
		\getwidthofnode{\sleft}{\x2}
		%draw line
		\draw[shorten <=-\sright/2, shorten >=-\sleft/2] ([yshift=-0.25cm]c\x1) -- ([yshift=-0.25cm]c\x2);
	} %else only one child
	{
		\getwidthofnode{\sleft}{\x}
		%get widht corresponding to the widht of the bigger node
		\pgfmathsetlength{\sright}{max(\sright,\sleft)}
		%draw line
		\draw (c\x)+(-\sright/2,+0.25) -- +(\sright/2,+0.25);
	}
}{}
}
\end{tikzpicture}
	\end{figure}
	For register one we still need $D(d)$.
	
	\begin{figure}[H]
		\centering
		\begin{tikzpicture}[grow=up,level distance=0.5cm,
execute at begin node=$, execute at end node=$,
every node/.style={opacity=1},
every child/.style={edge from parent/.style={opacity=0}}]
\def\dist{0.4cm}
\node(e) {\vdashf \falses} [sibling distance=2*\dist+4cm] 
	child {node(1) {\vdashf D(d)} [sibling distance=\dist+2.8cm]
		child[xshift=2cm] {node(11) {\vdashf R_1(b,d)}}
		child[xshift=2cm] {node(12) {\vdashf R_1(b,d)\to D(d)} [sibling distance=\dist+3.5cm]
			child[xshift=2cm] {node(121) {\vdashf S(a,b)}}
			child[xshift=2cm] {node(122) {\vdashf S(a,b)\to R_1(b,d)\to D(d)} [sibling distance=\dist+4.1cm]
				child[xshift=2cm] {node(1221) {\vdashf Q(a)}}
				child[xshift=2cm] {node(1222) {\vdashf Q(a)\to S(a,b)\to R_1(b,d)\to D(d)}
					child {node(12221) {\vdashf\forall\alpha\beta\delta(Q(\alpha)\to S(\alpha,\beta)\to R_1(\beta,\delta)\to D(\delta))}}}}}}
	child {node(2) {\vdashf D(d)\to\falses}
		child {node(21) {D(d)\vdashf\falses}
			child {node[label={[yshift=-0.3cm]above:\vdots}](211) {}}}};



%set of all positions in the tree
\def\nodes{,1,2,11,12,21,121,122,1221,1222,12221}
%the yshift does not work on nodes so we create a coordinate for every node
\foreach \x in \nodes{
\ifnodedefined{\x}{
	\coordinate(c\x) at (\x);
	\coordinate(5.\x) at (0,0);
}{}
}

%draw lines for each node with 2 children (add parents with 2 children)
\foreach \x in \nodes{
\ifnodedefined{5.\x1}{
	%calculate with of line
	\getwidthofnode{\sright}{\x1}
	%if there is a right child draw wide line
	\ifnodedefined{5.\x2}{
		\getwidthofnode{\sleft}{\x2}
		%draw line
		\draw[shorten <=-\sright/2, shorten >=-\sleft/2] ([yshift=-0.25cm]c\x1) -- ([yshift=-0.25cm]c\x2);
	} %else only one child
	{
		\getwidthofnode{\sleft}{\x}
		%get widht corresponding to the widht of the bigger node
		\pgfmathsetlength{\sright}{max(\sright,\sleft)}
		%draw line
		\draw (c\x)+(-\sright/2,+0.25) -- +(\sright/2,+0.25);
	}
}{}
}
\end{tikzpicture}
	\end{figure}
	Since register two should not change we only need $R_2(b,b_0)$.
	
	\begin{figure}[H]
		\centering
		\begin{tikzpicture}[grow=up,
execute at begin node=$, execute at end node=$,
every node/.style={opacity=1},
every child/.style={edge from parent/.style={opacity=0}}]
\def\dist{0.4cm}
\node(e) {\PModelsf \falses} [sibling distance=2*\dist+5cm] 
	child {node(1) {\PModelsf R_2(b,b_0)} [sibling distance=\dist+3.3cm]
		child[xshift=2cm] {node(11) {\PModelsf R_2(a,b_0)}}
		child[xshift=2cm] {node(12) {\PModelsf R_2(a,b_0)\to R_2(b,b_0)} [sibling distance=\dist+4cm]
			child[xshift=2cm] {node(121) {\PModelsf S(a,b)}}
			child[xshift=2cm] {node(122) {\PModelsf S(a,b)\to R_2(a,b_0)\to R_2(b,b_0)} [sibling distance=\dist+4.4cm]
				child[xshift=2cm] {node(1221) {\PModelsf Q(a)}}
				child[xshift=2cm] {node(1222) {\PModelsf Q(a)\to S(a,b)\to R_2(a,b_0)\to R_2(b,b_0)}
					child {node(12221) {\PModelsf\forall\alpha\beta\gamma(Q(\alpha)\to S(\alpha,\beta)\to R_2(\alpha,\gamma)\to R_2(\beta,\gamma))}}}}}}
	child {node(2) {\PModelsf R_2(b,b_0)\to\falses}
		child {node(21) {R_2(b,b_0)\PModelsf\falses}
			child {node[label={[yshift=-0.3cm]above:\vdots}](211) {}}}};


%set of all positions in the tree
\def\nodes{,1,2,11,12,21,121,122,1221,1222,12221,211}
\def\identifier{6}
\foreach \x in \nodes{
\ifnodedefined{\x}{
	\coordinate(c\x) at (\x);
	\coordinate(\identifier.\x) at (0,0);
}{}
}

%draw lines for each node with 2 children (add parents with 2 children)
\foreach \x in \nodes{
\ifnodedefined{\identifier.\x1}{
	%calculate with of line
	\getwidthofnode{\sright}{\x1}
	%if there is a right child draw wide line
	\ifnodedefined{\identifier.\x2}{
		\getwidthofnode{\sleft}{\x2}
		%draw line
		\draw[shorten <=-\sright/2, shorten >=-\sleft/2] ([yshift=-0.25cm]c\x1) -- ([yshift=-0.25cm]c\x2);
	} %else only one child
	{
		\getwidthofnode{\sleft}{\x}
		%get widht corresponding to the widht of the bigger node
		\pgfmathsetlength{\sright}{max(\sright,\sleft)}
		%draw line
		\draw (c\x)+(-\sright/2,+0.25) -- +(\sright/2,+0.25);
	}
}{}
}
\end{tikzpicture}
	\end{figure}
	%TODO
	Now we have $\Gamma_C$ (Since $P(a_{i-1},a_i)$ is already in $\Gamma_D$) and can deduce \false{} by induction hypothesis.\\
	\underline{Case $r=-(Q,1,Q_1,Q_2)$}
	%TODO note that r2 does not change, same for all cases
	\\\uline{$r_1=0$}
	
	\begin{figure}[H]
		\centering
		\begin{tikzpicture}[grow=up,level distance=0.5cm,
execute at begin node=$, execute at end node=$,
every node/.style={opacity=1},
every child/.style={edge from parent/.style={opacity=0}}]
\def\dist{0.4cm}
\node(e) {\vdashf \falses} [sibling distance=6cm] 
	child {node(1) {\vdashf Q_2(b)} [sibling distance=\dist+2.7cm]
		child[xshift=1.7cm] {node(11) {\vdashf E(a_0)}}
		child[xshift=1.7cm] {node(12) {\vdashf E(a_0)\to Q_2(b)} [sibling distance=\dist+3.8cm]
			child[xshift=2cm] {node(121) {\vdashf R_1(a,a_0)}}
			child[xshift=2cm] {node(122) {\vdashf R_1(a,a_0)\to E(a_0)\to Q_2(b)} [sibling distance=\dist+4.5cm]
				child[xshift=2cm] {node(1221) {\vdashf S(a,b)}}
				child[xshift=2cm] {node(1222) {\vdashf S(a,b)\to R_1(a,a_0)\to E(a_0)\to Q_2(b)} [sibling distance=\dist+5cm]
					child[xshift=2cm] {node(12221) {\vdashf Q(a)}}
					child[xshift=2cm] {node(12222) {\vdashf Q(a)\to S(a,b)\to R_1(a,a_0)\to E(a_0)\to Q_2(b)}
						child {node(122221) {\vdashf\forall\alpha\beta\gamma(Q(\alpha)\to S(\alpha,\beta)\to R_1(\alpha,\gamma)\to E(\gamma)\to Q_2(\beta))}}}}}}}
	child {node(2) {\vdashf Q_2(b)\to\falses}
		child {node(21) {Q_2(b)\vdashf\falses}
			child {node[label={[yshift=-0.3cm]above:\vdots}](211) {}}}};


%set of all positions in the tree
\def\nodes{,1,2,11,12,21,121,122,1221,1222,12222,12221,12222,122221}
%the yshift does not work on nodes so we create a coordinate for every node
\foreach \x in \nodes{
\ifnodedefined{\x}{
	\coordinate(c\x) at (\x);
	\coordinate(7.\x) at (0,0);
}{}
}

%draw lines for each node with 2 children (add parents with 2 children)
\foreach \x in \nodes{
\ifnodedefined{7.\x1}{
	%calculate with of line
	\getwidthofnode{\sright}{\x1}
	%if there is a right child draw wide line
	\ifnodedefined{7.\x2}{
		\getwidthofnode{\sleft}{\x2}
		%draw line
		\draw[shorten <=-\sright/2, shorten >=-\sleft/2] ([yshift=-0.25cm]c\x1) -- ([yshift=-0.25cm]c\x2);
	} %else only one child
	{
		\getwidthofnode{\sleft}{\x}
		%get widht corresponding to the widht of the bigger node
		\pgfmathsetlength{\sright}{max(\sright,\sleft)}
		%draw line
		\draw (c\x)+(-\sright/2,+0.25) -- +(\sright/2,+0.25);
	}
}{}
}
\end{tikzpicture}
	\end{figure}
	
	$r_1$ stays zero
	
	\begin{figure}[H]
		\centering
		\begin{tikzpicture}[grow=up,level distance=0.5cm,
execute at begin node=$, execute at end node=$,
every node/.style={opacity=1},
every child/.style={edge from parent/.style={opacity=0}}]
\node(e) {\PModelsf \falses} [sibling distance=6.4cm] 
	child {node(1) {\PModelsf R_1(b,a_0)} [sibling distance=3.2cm]
		child[xshift=1.7cm] {node(11) {\PModelsf E(a_0)}}
		child[xshift=1.7cm] {node(12) {\PModelsf E(a_0)\to R_1(b,a_0)} [sibling distance=4.5cm]
			child[xshift=2cm] {node(121) {\PModelsf R_1(a,a_0)}}
			child[xshift=2cm] {node(122) {\PModelsf R_1(a,a_0)\to E(a_0)\to R_1(b,a_0)} [sibling distance=5.2cm]
				child[xshift=2cm] {node(1221) {\PModelsf S(a,b)}}
				child[xshift=2cm] {node(1222) {\PModelsf S(a,b)\to R_1(a,a_0)\to E(a_0)\to R_1(b,a_0)} [sibling distance=5.9cm]
					child[xshift=2cm] {node(12221) {\PModelsf Q(a)}}
					child[xshift=2cm] {node(12222) {\PModelsf Q(a)\to S(a,b)\to R_1(a,a_0)\to E(a_0)\to R_1(b,a_0)}
						child {node(122221) {\PModelsf\forall\alpha\beta\gamma(Q(\alpha)\to S(\alpha,\beta)\to R_1(\alpha,\gamma)\to E(\gamma)\to R_1(\beta,\gamma))}}}}}}}
	child {node(2) {\PModelsf R_1(b,a_0)\to\falses}
		child {node(21) {R_1(b,a_0)\PModelsf\falses}
			child {node[label={[yshift=-0.3cm]above:\vdots}](211) {}}}};


%set of all positions in the tree
\def\nodes{,1,2,11,12,21,121,122,1221,1222,12222,12221,12222,122221,211}
\def\identifier{8}
\foreach \x in \nodes{
\ifnodedefined{\x}{
	\coordinate(c\x) at (\x);
	\coordinate(\identifier.\x) at (0,0);
}{}
}

%draw lines for each node with 2 children (add parents with 2 children)
\foreach \x in \nodes{
\ifnodedefined{\identifier.\x1}{
	%calculate with of line
	\getwidthofnode{\sright}{\x1}
	%if there is a right child draw wide line
	\ifnodedefined{\identifier.\x2}{
		\getwidthofnode{\sleft}{\x2}
		%draw line
		\draw[shorten <=-\sright/2, shorten >=-\sleft/2] ([yshift=-0.25cm]c\x1) -- ([yshift=-0.25cm]c\x2);
	} %else only one child
	{
		\getwidthofnode{\sleft}{\x}
		%get widht corresponding to the widht of the bigger node
		\pgfmathsetlength{\sright}{max(\sright,\sleft)}
		%draw line
		\draw (c\x)+(-\sright/2,+0.25) -- +(\sright/2,+0.25);
	}
}{}
}
\end{tikzpicture}
	\end{figure}
	\uline{$r_1\geq1$}
	
	new state $Q_1$
	
	\begin{figure}[H]
		\centering
		\begin{tikzpicture}[grow=up,level distance=0.5cm,
execute at begin node=$, execute at end node=$,
every node/.style={opacity=1},
every child/.style={edge from parent/.style={opacity=0}}]
\node(e) {\vdashf \falses} [sibling distance=6.4cm] 
	child {node(1) {\vdashf Q_1(b)} [sibling distance=3.2cm]
		child[xshift=1.7cm] {node(11) {\vdashf D(a_0)}}
		child[xshift=1.7cm] {node(12) {\vdashf D(a_0)\to Q_1(b)} [sibling distance=4.5cm]
			child[xshift=2cm] {node(121) {\vdashf R_1(a,a_0)}}
			child[xshift=2cm] {node(122) {\vdashf R_1(a,a_0)\to D(a_0)\to Q_1(b)} [sibling distance=5.2cm]
				child[xshift=2cm] {node(1221) {\vdashf S(a,b)}}
				child[xshift=2cm] {node(1222) {\vdashf S(a,b)\to R_1(a,a_0)\to D(a_0)\to Q_1(b)} [sibling distance=5.9cm]
					child[xshift=2cm] {node(12221) {\vdashf Q(a)}}
					child[xshift=2cm] {node(12222) {\vdashf Q(a)\to S(a,b)\to R_1(a,a_0)\to D(a_0)\to Q_1(b)}
						child {node(122221) {\vdashf\forall\alpha\beta\gamma(Q(\alpha)\to S(\alpha,\beta)\to R_1(\alpha,\gamma)\to D(\gamma)\to Q_1(\beta))}}}}}}}
	child {node(2) {\vdashf Q_1(b)\to\falses}
		child {node(21) {Q_1(b)\vdashf\falses}
			child {node[label={[yshift=-0.3cm]above:\vdots}](211) {}}}};



%set of all positions in the tree
\def\nodes{,1,2,11,12,21,121,122,1221,1222,12222,12221,12222,122221}
%the yshift does not work on nodes so we create a coordinate for every node
\foreach \x in \nodes{
\ifnodedefined{\x}{
	\coordinate(c\x) at (\x);
	\coordinate(9.\x) at (0,0);
}{}
}

%draw lines for each node with 2 children (add parents with 2 children)
\foreach \x in \nodes{
\ifnodedefined{9.\x1}{
	%calculate with of line
	\getwidthofnode{\sright}{\x1}
	%if there is a right child draw wide line
	\ifnodedefined{9.\x2}{
		\getwidthofnode{\sleft}{\x2}
		%draw line
		\draw[shorten <=-\sright/2, shorten >=-\sleft/2] ([yshift=-0.25cm]c\x1) -- ([yshift=-0.25cm]c\x2);
	} %else only one child
	{
		\getwidthofnode{\sleft}{\x}
		%get widht corresponding to the widht of the bigger node
		\pgfmathsetlength{\sright}{max(\sright,\sleft)}
		%draw line
		\draw (c\x)+(-\sright/2,+0.25) -- +(\sright/2,+0.25);
	}
}{}
}
\end{tikzpicture}
	\end{figure}
	
	decrement $r_1$
	
	\begin{figure}[H]
		\centering
		\begin{tikzpicture}[grow=up,level distance=0.5cm,
execute at begin node=$, execute at end node=$,
every node/.style={opacity=1},
every child/.style={edge from parent/.style={opacity=0}}]
\def\dist{0.7cm}
\node(e) {\vdashf \falses} [sibling distance=2*\dist+6.4cm] 
	child {node(1) {\vdashf R_1(b,a_1)} [sibling distance=\dist+2.9cm]
		child[xshift=1.7cm] {node(11) {\vdashf P(a_0,a_1)}}
		child[xshift=1.7cm] {node(12) {\vdashf P(a_0,a_1)\to R_1(b,a_1)} [sibling distance=\dist+3.5cm]
			child[xshift=1.7cm] {node(121) {\vdashf D(a_0)}}
			child[xshift=1.7cm] {node(122) {\vdashf D(a_0)\to P(a_0,a_1)\to R_1(b,a_1)} [sibling distance=\dist+4.5cm]
				child[xshift=2cm] {node(1221) {\vdashf R_1(a,a_0)}}
				child[xshift=2cm] {node(1222) {\vdashf R_1(a,a_0)\to D(a_0)\to P(a_0,a_1)\to R_1(b,a_1)} [sibling distance=\dist+5.2cm]
					child[xshift=2cm] {node(12221) {\vdashf S(a,b)}}
					child[xshift=2cm] {node(12222) {\vdashf S(a,b)\to R_1(a,a_0)\to D(a_0)\to P(a_0,a_1)\to R_1(b,a_1)} [sibling distance=\dist+5.9cm]
						child[xshift=2cm] {node(122221) {\vdashf Q(a)}}
						child[xshift=2cm] {node(122222) {\vdashf Q(a)\to S(a,b)\to R_1(a,a_0)\to D(a_0)\to P(a_0,a_1)\to R_1(b,a_1)}
							child {node(1222221) {\vdashf\forall\alpha\beta\gamma\delta(Q(\alpha)\to S(\alpha,\beta)\to R_1(\alpha,\gamma)\to D(\gamma) \to P(\gamma,\delta)\to R_1(\beta,\delta))}}}}}}}}
	child {node(2) {\vdashf R_1(b,a_1)\to\falses}
		child {node(21) {R_1(b,a_1)\vdashf\falses}
			child {node[label={[yshift=-0.3cm]above:\vdots}](211) {}}}};


%tikz did not want to do this in the loop no idea why
%\coordinate(c1222) at (1222);
%\getwidthofnode{\sright}{12221}
%\draw (c1222)+(-\sright/2,+0.25) -- +(\sright/2,+0.25);

%set of all positions in the tree
\def\nodes{,1,2,11,12,21,121,122,1221,1222,12222,12221,122222,122221,122222,1222221,211}
%the yshift does not work on nodes so we create a coordinate for every node
\foreach \x in \nodes{
\ifnodedefined{\x}{
	\coordinate(c\x) at (\x);
	\coordinate(10.\x) at (0,0);
}{}
}

%draw lines for each node with 2 children (add parents with 2 children)
\foreach \x in \nodes{
\ifnodedefined{10.\x1}{
	%calculate with of line
	\getwidthofnode{\sright}{\x1}
	%if there is a right child draw wide line
	\ifnodedefined{10.\x2}{
		\getwidthofnode{\sleft}{\x2}
		%draw line
		\draw[shorten <=-\sright/2, shorten >=-\sleft/2] ([yshift=-0.25cm]c\x1) -- ([yshift=-0.25cm]c\x2);
	} %else only one child
	{
		\getwidthofnode{\sleft}{\x}
		%get widht corresponding to the widht of the bigger node
		\pgfmathsetlength{\sright}{max(\sright,\sleft)}
		%draw line
		\draw (c\x)+(-\sright/2,+0.25) -- +(\sright/2,+0.25);
	}
}{}
}
\end{tikzpicture}
	\end{figure}
	
\end{proof}

\begin{lemma}\label{lem.19}
	\begin{align*} %TODO for any M one can effectively construct? or is M the M from the construction
		  & \text{$M$ terminates on input $(0,0)$} &   & \text{iff} & \text{$\Gamma_M\vdash\false$ holds in system P.} 
	\end{align*}
\end{lemma}
\begin{proof}
	The $\Leftarrow$ directions follows directly from Claim \ref{cla.17}. And the $\Rightarrow$ direction is a direct consequence of Claim \ref{cla.18} with $C=\langle Q_0,0,0\rangle$.
\end{proof}

\begin{theorem}
	\PCons{} is undecidable.
\end{theorem}
\begin{proof}
	Since by Lemma \ref{lem.19} for a given two-counter automaton $M$ we can effectively construct a set of \SysP-formulas $\Gamma_M$ such that $M$ terminates on input $(0,0)$ iff $\Gamma_M$ is not consistent. It follows that $\autHalt\leq\PCons$. Hence, since \autHalt{} is undecidable, we have shown that \PCons{} is undecidable too.
\end{proof}
