\section{System P}
\subsection{Definitions}
Let $\VarP=\{\alpha,\beta,\dots\}$ be a countably infinite set (of variables) and $\RelP=\{\false^{(0)},P^{(2)},Q^{(2)},\dots\}$ a ranked alphabet (of relation symbols).
A first-order logic formula $\varphi$ is an
\begin{description} %TODO \overrightarrow{\alpha}
\item[atomic formula] if $\varphi=\false$ or $\varphi=P(\alpha,\beta)$ for some $P\in\RelP$ and $\alpha,\beta\in\VarP$.
\item[universal formula] if $\varphi=\forall\overrightarrow{\alpha}(A_1\to A_2 \to\dots\to A_n)$ where $A_i$ is an atomic formula for $i\in\left[n\right]$, $A_i\neq\false$ for $i\in\left[n-1\right]$ and for each $\alpha\in\FV(\varphi)\cap\FV(A_n)$ there exists an $i\in\left[n-1\right]$ such that $\alpha\in\FV(A_i)$.
\item[existential formula] if there exits $n\ge0$,  atomic formulas $A_i\neq\false$ for $i\in\left[n\right]$ such that $\varphi=\forall\overrightarrow{\alpha}(A_1\to A_2 \to\dots\to A_{n-1}\to\forall\beta(A_n\to\false)\to\false)$.
\end{description}
The set of formulas of System \SysP{} over \VarP{} and \RelP{} is the set of all first order formulas over the same "alphabet" that are either an atomic, universal or existential formula.\\ 
$\FV(\Gamma)=\bigcup\{\FV(A)\mid A\in \Gamma\}$\\
Deduction Rules
\begin{mdframed}
\begingroup
\addtolength{\jot}{0.3cm}
\begin{align*}
&(\text{Axiom}) &&\Gamma,A\vdash A\\
&(\rightarrow\text{-Introduction}) &&\frac{\Gamma,A\vdash B}{\Gamma\vdash A\to B}\\
&(\rightarrow\text{-Elimination}) &&\frac{\Gamma\vdash A\to B \hspace{0.4cm}\Gamma\vdash A}{\Gamma\vdash B}\\
&(\forall\text{-Introduction}) &&\frac{\Gamma\vdash B}{\Gamma\vdash \forall\alpha B} &&\alpha\notin\emph{FV}(\Gamma) \\
&(\forall\text{-Elimination}) &&\frac{\Gamma\vdash \forall\alpha B }{\Gamma\vdash B\left[ \alpha:=b\right] } %TODO dot after \alpha?
\end{align*}
\endgroup
\end{mdframed}
An Interpretation $I$ of a P formula is a tuple $I=(\Delta,\cdot^I)$ where $\Delta$ is a set (called domain), $P^I\subseteq\Delta^k$ and $\alpha^I\in\Delta$.\dots\\
If we interpret \false{} with the logical constant false ($\bot$) (denoted by $\vdash_f$) we can add a new deduction rule.
\begin{mdframed}
\begin{align*}
&(\exists\text{-Introduction}) &&\frac
{\Gamma,A\left[\alpha:=a\right]\vdashf B}
{\Gamma,\forall\alpha(A\to\false)\to\false\vdashf B} && a\notin\emph{FV}(\Gamma,A,B)
\end{align*}
\end{mdframed}
\begin{proof}
Let $I=(\Delta,\cdot^I)$ be a model of $\Gamma,\forall\alpha(A\to\false)\to\false$ with $\false^I=\bot$.
\begin{align*}
I\models\Gamma,\forall\alpha(A\to\false)\to\false&\Rightarrow I\models\forall\alpha(A\to\false)\to\false\\
&\Rightarrow (\forall\alpha(A\to\false))^I\to\false^I\\
&\Rightarrow (\forall\alpha(A\to\false))^I\to\bot\\
&\Rightarrow \neg(\forall\alpha(A\to\false))^I\\
&\Rightarrow \neg(\forall d\in\Delta:(A\to\false)^{I\left[\alpha\mapsto d\right]})\\
&\Rightarrow \exists d\in\Delta:\neg(A^{I\left[\alpha\mapsto d\right]}\to\false^{I\left[\alpha\mapsto d\right]})\\
&\Rightarrow \exists d\in\Delta:\neg(A^{I\left[\alpha\mapsto d\right]}\to\bot)\\
&\Rightarrow \exists d\in\Delta:\neg(\neg A^{I\left[\alpha\mapsto d\right]})\\
&\Rightarrow \exists d\in\Delta:A^{I\left[\alpha\mapsto d\right]}\\
\end{align*}
Together with $a\notin\emph{FV}(\Gamma,A)$, it follows that $I\left[a\mapsto d\right]$ is a model of $\Gamma,A\left[\alpha:=a\right]$. Which implies $I\left[a\mapsto d\right]\models B$.	Since $a$ is not free in $B$ we conclude that $I$ is also a model of $B$.
\end{proof}
\subsection{Provability in System P is undecidable}
$\Gamma_C:$
\begin{itemize}
\item $Q(a)$
\item $R_1(a,a_0),P(a_{i-1},a_i)~\text{for}~i\in\{1,\dots,m\}$
\item $R_2(a,b_0),P(b_{i-1},b_i)~\text{for}~i\in\{1,\dots,n\}$
\item $D(a),D(a_i),D(b_j)~\text{for}~i\in\{1,\dots,m\}~\text{and}~j\in\{1,\dots,n\}$
\item $E(a_m),E(b_n)$
\end{itemize}

$+(Q,1,Q'):$
\begin{itemize}
\item $\forall\alpha\beta(Q(\alpha)\to S(\alpha,\beta)\to Q'(\beta))$ \\change of state
\item $\forall\alpha\beta\gamma\delta(Q(\alpha)\to S(\alpha,\beta)\to R_1(\alpha,\gamma)\to R_1(\beta,\delta)\to P(\delta,\gamma))$\\increment register 1
\item $\forall\alpha\beta\delta(Q(\alpha)\to S(\alpha,\beta)\to R_1(\beta,\delta)\to D(\delta))$ \\prevent zero
\item $\forall\alpha\beta\gamma(Q(\alpha)\to S(\alpha,\beta)\to R_2(\alpha,\gamma)\to R_2(\beta,\gamma))$ \\do not change register 2
\end{itemize}

$-(Q,1,Q_1,Q_2):$
\begin{itemize}
\item $\forall\alpha\beta\gamma(Q(\alpha)\to S(\alpha,\beta)\to R_1(\alpha,\gamma)\to E(\gamma)\to Q_2(\beta))$\\jump on zero
\item $\forall\alpha\beta\gamma(Q(\alpha)\to S(\alpha,\beta)\to R_1(\alpha,\gamma)\to E(\gamma)\to R_1(\beta,\gamma)$\\register 1 stays zero
\item $\forall\alpha\beta\gamma(Q(\alpha)\to S(\alpha,\beta)\to R_1(\alpha,\gamma)\to D(\gamma)\to Q_1(\beta))$\\change state if register 1 is greater zero
\item $\forall\alpha\beta\gamma\delta(Q(\alpha)\to S(\alpha,\beta)\to R_1(\alpha,\gamma)\to D(\gamma) \to P(\gamma,\delta)\to R_1(\beta,\delta))$\\decrement register 1
\item $\forall\alpha\beta\gamma(Q(\alpha)\to S(\alpha,\beta)\to R_2(\alpha,\gamma)\to R_2(\beta,\gamma))$\\do not change register 2
\end{itemize}
\begin{lemma}
\begin{align*}
&\text{$M$ terminates on input $(0,0)$}&& \text{iff} &\text{$\Gamma_M\vdash\false$ holds in system P.}
\end{align*}
\end{lemma}
\begin{claim}
\begin{align*}
&\text{$\Gamma_M\vdash\false$ holds in system P} && \implies &\text{$M$ terminates on input $(0,0)$}
\end{align*}
\end{claim}
\begin{proof}
Assume $M$ does not terminate then there is an infinite chain $C_1\Rightarrow_M C_2\Rightarrow_M C_3\Rightarrow_M\dots$. ($C_i=\langle Q_i,m_i,n_i\rangle$) Now we construct a model of $\Gamma_M$ which interprets \false{} with $\bot$ this contradicts $\Gamma_M\vdash\false$.

To illustrate the idea we will use a graphical notation for an interpretation $I$.
By
\tikz[baseline=-3pt]{
\node (1) {$d_1$};
\node[right of=1] (2) {$d_2$};
\path[->,>=latex'] (1) edge node[above,scale=0.8] {R} (2);}
we say that $(d_1,d_2)\in R^I$. And we use
\tikz[baseline=5pt]{
\node (1) {$d$};
\node[node distance = 0.6cm,above of=1,scale=0.8] {P};}
to say that $d\in P^I$ for unary predicate symbols. Now the idea for our model of $\Gamma_M$ looks like this:

\begin{figure}[H]
\begin{tikzpicture}[execute at begin node=$, execute at end node=$,
label/.style={scale=0.8},
every path/.style={->,>=latex'},node distance = 1.4cm]
\node (0) {0};
\node[right of=0] (1) {1};
\node[right of=1] (2) {\dots};
\node[right of=2] (3) {m_i};
\node[right of=3] (4) {\dots};
\node[right of=4] (5) {n_i};
\node[right of=5] (6) {\dots};
\node[right of=6] (7) {i};
\node[right of=7] (8) {\dots};


\node[node distance = 0.6cm,above of=1,label]     {Q_1};
\node[node distance = 0.6cm,above of=3,label]     {Q_{m_i}};
\node[node distance = 0.6cm,above of=5,label]     {Q_{n_i}};
\node[node distance = 0.6cm,above of=7,label]     {Q_i};


\foreach \x in {0,...,7}{
\pgfmathtruncatemacro{\xn}{\x+1}
\path (\xn) edge[bend left=10] node[below,label] {P} (\x);
\path (\x) edge[bend left=10] node[above,label] {S} (\xn);
}

\path (1) edge[bend right=60] node[above,label] {R_1} (0);
\path (1) edge[bend left=60] node[below,label] {R_2} (0);

\path (7) edge[bend right=50] node[above,label] {R_1} (3);
\path (7) edge[bend left=60] node[below,label] {R_2} (5);


\end{tikzpicture}
\end{figure}
We have $0\in E^I$ and all other numbers are in $D^I$.

Here is the more formal definition of our model $I=(\mathbb{N},\cdot^I)$.
\begin{align*}
& P^I=\{(i+1,i)\mid i\in\mathbb{N}\} && R_1^I=\{(i,m_i)\mid i\in\mathbb{N}\} && R_2^I=\{(i,n_i)\mid i\in\mathbb{N}\}\\
& Q^I=\{ i\in\mathbb{N}\setminus\{0\}\mid Q=Q_i\} && D^I=\mathbb{N}\setminus\{0\} && E^I=\{0\}\\
& S^I=\{(i,i+1)\mid i\in\mathbb{N}\}
\end{align*}
\begin{align*}
& a^I=1 && a_0^I=0 && b_0^I=0
\end{align*}
\end{proof}
\begin{claim}
If a final state is reachable from $C$ then $\Gamma_C\cup\Gamma\vdash\false$.
\end{claim}
\begin{proof} By induction on the length of the computation.
For the tableau proofs we will abbreviate \false{} by \falses.\\
Induction Base trivial \dots\\
Induction Step\\
$C\Rightarrow_M^r D$\\
We need to make a case distinction on the rule $r$.\\
\underline{Case $r=+(Q,1,Q')$}\\
Basic idea:\\
\[
\cfrac
{\cfrac{IH}{\Gamma_C\cup\Gamma\cup\Gamma_D\vdash\falses}\hspace{0.6cm}
\cfrac{}{\Gamma_C\cup\Gamma\vdash\Gamma_D}}
{\Gamma_C\cup\Gamma\vdash\falses}
\]
Since $I\models\false$ holds trivially if $I$ interprets \false{} with $\top$ we only need to consider models (note that there are none if $M$ terminates which is exactly what we want to proof) of $\Gamma_C\cup\Gamma$ that interpret \false{} with $\bot$ (so we can use our new deduction rule).\\
We will just drop $\Gamma_C\cup\Gamma$ and only write new formulas on the left side.\\
We first introduce the new variables needed for $\Gamma_D$ (let $b,d\in \VarP\setminus\text{FV}(\Gamma_C\cup\Gamma)$). Intuitively $b$ will represent the successor state and $d$ will be the anchor for register one.

\begin{figure}[H]
\begin{tikzpicture}[grow=up,level distance=0.5cm,
execute at begin node=$, execute at end node=$,
every node/.style={opacity=1},
every child/.style={edge from parent/.style={opacity=0}}]
\node(e) {\PModelsf \falses} [sibling distance=8cm] 
	child [sibling distance=4cm] {node(1) {\PModelsf\forall\beta(S(a,\beta)\to\falses)\to\falses} 
		child {node(11) {\PModelsf\forall\alpha(\forall\beta(S(\alpha,\beta)\to\falses)\to\falses)}}}
	child {node(2) {\PModelsf(\forall\beta(S(a,\beta)\to\falses)\to\falses)\to\falses}
		child {node(21) {\forall\beta(S(a,\beta)\to\falses)\to\falses\PModelsf\falses}
			child {node(211) {S(a,b)\PModelsf\falses}
				child {node[label={[yshift=-0.3cm]above:\vdots}](2111) {}}}}};

%set of all positions in the tree
\def\nodes{,1,2,11,21,211,2111}
%the yshift does not work on nodes so we create a coordinate for every node
\foreach \x in \nodes{
\ifnodedefined{\x}{
	\coordinate(c\x) at (\x);
	\coordinate(1.\x) at (0,0); %we need this because defined nodes are not fergotten between tikz pictures
}{}
}

%draw lines for each node with 2 children (add parents with 2 children)
\foreach \x in \nodes{
\ifnodedefined{1.\x1}{
	%calculate with of line
	\getwidthofnode{\sright}{\x1}
	%if there is a right child draw wide line
	\ifnodedefined{1.\x2}{
		\getwidthofnode{\sleft}{\x2}
		%draw line
		\draw[shorten <=-\sright/2, shorten >=-\sleft/2] ([yshift=-0.25cm]c\x1) -- ([yshift=-0.25cm]c\x2);
	} %else only one child
	{
		\getwidthofnode{\sleft}{\x}
		%get widht corresponding to the widht of the bigger node
		\pgfmathsetlength{\sright}{max(\sright,\sleft)}
		%draw line
		\draw (c\x)+(-\sright/2,+0.25) -- +(\sright/2,+0.25);
	}
}{}
}

\end{tikzpicture}
\end{figure}
The formula $R_1(b,d)$ can be acquired in a similar way. Again we will just drop $S(a,b)$ and $D(b)$ on the left side for comprehensibility.

\begin{figure}[H]
\begin{tikzpicture}[grow=up,level distance=0.5cm,
execute at begin node=$, execute at end node=$,
every node/.style={opacity=1},
every child/.style={edge from parent/.style={opacity=0}}]
\node(e) {\PModelsf\falses} [sibling distance=6.5cm]
	child {node(1) {\PModelsf\forall\beta(R_1(b,\beta)\to\falses)\to\falses} [sibling distance=4cm]
		child {node(11) {\PModelsf\forall\alpha(\forall\beta(R_1(\alpha,\beta)\to\falses)\to\falses)}}}
	child {node(2) {\PModelsf(\forall\beta(R_1(b,\beta)\to\falses)\to\falses)\to\falses}
		child {node(21) {\forall\beta(R_1(b,\beta)\to\falses)\to\falses\PModelsf\falses}
			child {node(211) {R_1(b,d)\PModelsf\falses}
				child {node[label={[yshift=-0.3cm]above:\vdots}](2111) {}}}}};

%set of all positions in the tree
\def\nodes{,1,2,11,21,211,2111}
%the yshift does not work on nodes so we create a coordinate for every node
\foreach \x in \nodes{
\ifnodedefined{\x}{
	\coordinate(c\x) at (\x);
	\coordinate(2.\x) at (0,0);
}{}
}

%draw lines for each node with 2 children (add parents with 2 children)
\foreach \x in \nodes{
\ifnodedefined{2.\x1}{
	%calculate with of line
	\getwidthofnode{\sright}{\x1}
	%if there is a right child draw wide line
	\ifnodedefined{2.\x2}{
		\getwidthofnode{\sleft}{\x2}
		%draw line
		\draw[shorten <=-\sright/2, shorten >=-\sleft/2] ([yshift=-0.25cm]c\x1) -- ([yshift=-0.25cm]c\x2);
	} %else only one child
	{
		\getwidthofnode{\sleft}{\x}
		%get widht corresponding to the widht of the bigger node
		\pgfmathsetlength{\sright}{max(\sright,\sleft)}
		%draw line
		\draw (c\x)+(-\sright/2,+0.25) -- +(\sright/2,+0.25);
	}
}{}
}

\end{tikzpicture}
\end{figure}
Now we have all the new free variables we need and we continue by ensuring that these variables fulfill all the formulas in $\Gamma_D$.

\begin{figure}[H]
\begin{tikzpicture}[grow=up,level distance=0.5cm,
execute at begin node=$, execute at end node=$,
every node/.style={opacity=1},
every child/.style={edge from parent/.style={opacity=0}}]
\node(e) {\vdashf \falses} [sibling distance=12.4cm] 
	child [sibling distance=4cm] {node(1) {\vdashf Q'(b)} 
		child {node(11) {\vdashf S(a,b)} }
		child [sibling distance=3.7cm] {node(12) {\vdashf S(a,b)\to Q'(b)} 
			child {node(121) {\vdashf Q(a)} }
			child {node(122) {\vdashf Q(a)\to S(a,b)\to Q'(b)} 
				child {node(1221) {\vdashf\forall\alpha\beta(Q(\alpha)\to S(\alpha,\beta)\to Q'(\beta))} }}}}
	child {node(2) {\vdashf Q'(b)\to\falses} 
		child {node(21) {Q'(b)\vdashf\falses} 
			child {node[label={[yshift=-0.3cm]above:\vdots}](211) {}}}};

%set of all positions in the tree
\def\nodes{,1,2,11,12,21,121,122,1221,211}
%the yshift does not work on nodes so we create a coordinate for every node
\foreach \x in \nodes{
\ifnodedefined{\x}{
	\coordinate(c\x) at (\x);
	\coordinate(3.\x) at (0,0);
}{}
}

%draw lines for each node with 2 children (add parents with 2 children)
\foreach \x in \nodes{
\ifnodedefined{3.\x1}{
	%calculate with of line
	\getwidthofnode{\sright}{\x1}
	%if there is a right child draw wide line
	\ifnodedefined{3.\x2}{
		\getwidthofnode{\sleft}{\x2}
		%draw line
		\draw[shorten <=-\sright/2, shorten >=-\sleft/2] ([yshift=-0.25cm]c\x1) -- ([yshift=-0.25cm]c\x2);
	} %else only one child
	{
		\getwidthofnode{\sleft}{\x}
		%get widht corresponding to the widht of the bigger node
		\pgfmathsetlength{\sright}{max(\sright,\sleft)}
		%draw line
		\draw (c\x)+(-\sright/2,+0.25) -- +(\sright/2,+0.25);
	}
}{}
}
\end{tikzpicture}
\end{figure}
Starting from $Q'(b)\vdashf\false$ we can connect $d$ and $a_0$.

\begin{figure}[H]
\begin{tikzpicture}[grow=up,level distance=0.5cm,
execute at begin node=$, execute at end node=$,
every node/.style={opacity=1},
every child/.style={edge from parent/.style={opacity=0}}]
\def\dist{0.4cm}
\node(e) {\PModelsf \falses} [sibling distance=2*\dist+8cm] 
	child [sibling distance=\dist+3cm] {node(1) {\PModelsf P(d,a_0)} 
		child {node(11) {\PModelsf R_1(b,d)} }
		child [sibling distance=\dist+3.9cm] {node(12) {\PModelsf R_1(b,d)\to P(d,a_0)} 
			child {node(121) {\PModelsf R_1(a,a_0)} }
			child [sibling distance=\dist+4.5cm] {node(122) {\PModelsf R_1(a,a_0)\to R_1(b,d)\to P(d,a_0)} 
				child {node(1221) {\PModelsf S(a,b)} }
				child [sibling distance=\dist+5cm] {node(1222) {\PModelsf S(a,b)\to R_1(a,a_0)\to R_1(b,d)\to P(d,a_0)}
					child {node(12221) {\PModelsf Q(a)} }
					child {node(12222) {\PModelsf Q(a)\to S(a,b)\to R_1(a,a_0)\to R_1(b,d)\to P(d,a_0)}
						child {node(122221) {\PModelsf \forall\alpha\beta\gamma\delta(Q(\alpha)\to S(\alpha,\beta)\to R_1(\alpha,\gamma)\to R_1(\beta,\delta)\to P(\delta,\gamma))}}}
					child {}}
				child {}}
			child {}}
		child {}}
				%child {node(1221) {\PModelsf\forall\alpha\beta(Q(\alpha)\to S(\alpha,\beta)\to Q'(\beta))} }}}}
	child {node(2) {\PModelsf P(d,a_0)\to\falses} 
		child {node(21) {P(d,a_0)\PModelsf\falses} 
			child {node[label={[yshift=-0.3cm]above:\vdots}](211) {}}}};

%set of all positions in the tree
\def\nodes{,1,2,11,12,21,121,122,1221,1222,12221,12222,122221,211}
%the yshift does not work on nodes so we create a coordinate for every node
\foreach \x in \nodes{
\ifnodedefined{\x}{
	\coordinate(c\x) at (\x);
	\coordinate(4.\x) at (0,0);
}{}
}

%draw lines for each node with 2 children (add parents with 2 children)
\foreach \x in \nodes{
\ifnodedefined{4.\x1}{
	%calculate with of line
	\getwidthofnode{\sright}{\x1}
	%if there is a right child draw wide line
	\ifnodedefined{4.\x2}{
		\getwidthofnode{\sleft}{\x2}
		%draw line
		\draw[shorten <=-\sright/2, shorten >=-\sleft/2] ([yshift=-0.25cm]c\x1) -- ([yshift=-0.25cm]c\x2);
	} %else only one child
	{
		\getwidthofnode{\sleft}{\x}
		%get widht corresponding to the widht of the bigger node
		\pgfmathsetlength{\sright}{max(\sright,\sleft)}
		%draw line
		\draw (c\x)+(-\sright/2,+0.25) -- +(\sright/2,+0.25);
	}
}{}
}
\end{tikzpicture}
\end{figure}
$R_2(b,b_0)$ can be deduced in the same way.\\
Now we have $\Gamma_C$ (Since $P(a_{i-1},a_i)$ is already in $\Gamma_D$) and can deduce \false{} by induction hypothesis.\\
\underline{Case $r=-(Q,1,Q_1,Q_2)$}
\end{proof}