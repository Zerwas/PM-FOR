\begin{tikzpicture}[grow=up,level distance=0.5cm,
execute at begin node=$, execute at end node=$,
every node/.style={opacity=1},
every child/.style={edge from parent/.style={opacity=0}}]
\node(e) {\vdashf \falses} [sibling distance=12.4cm] 
	child [sibling distance=4cm] {node(1) {\vdashf Q'(b)} 
		child {node(11) {\vdashf S(a,b)} }
		child [sibling distance=3.7cm] {node(12) {\vdashf S(a,b)\to Q'(b)} 
			child {node(121) {\vdashf Q(a)} }
			child {node(122) {\vdashf Q(a)\to S(a,b)\to Q'(b)} 
				child {node(1221) {\vdashf\forall\alpha\beta(Q(\alpha)\to S(\alpha,\beta)\to Q'(\beta))} }}}}
	child {node(2) {\vdashf Q'(b)\to\falses} 
		child {node(21) {Q'(b)\vdashf\falses} 
			child {node[label={[yshift=-0.3cm]above:\vdots}](211) {}}}};

%set of all positions in the tree
\def\nodes{,1,2,11,12,21,121,122,1221,211}
%the yshift does not work on nodes so we create a coordinate for every node
\foreach \x in \nodes{
\ifnodedefined{\x}{
	\coordinate(c\x) at (\x);
	\coordinate(3.\x) at (0,0);
}{}
}

%draw lines for each node with 2 children (add parents with 2 children)
\foreach \x in \nodes{
\ifnodedefined{3.\x1}{
	%calculate with of line
	\getwidthofnode{\sright}{\x1}
	%if there is a right child draw wide line
	\ifnodedefined{3.\x2}{
		\getwidthofnode{\sleft}{\x2}
		%draw line
		\draw[shorten <=-\sright/2, shorten >=-\sleft/2] ([yshift=-0.25cm]c\x1) -- ([yshift=-0.25cm]c\x2);
	} %else only one child
	{
		\getwidthofnode{\sleft}{\x}
		%get widht corresponding to the widht of the bigger node
		\pgfmathsetlength{\sright}{max(\sright,\sleft)}
		%draw line
		\draw (c\x)+(-\sright/2,+0.25) -- +(\sright/2,+0.25);
	}
}{}
}
\end{tikzpicture}