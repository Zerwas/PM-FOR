\section{Basic Definitions}
We will denote the set $\{1,\dots,n\}$ by $\left[n\right]$.
\begin{definition}
A \uline{ranked set} is a tuple $(\Sigma,\rank)$, where $\Sigma$ is a countable set and $\rank:\Sigma\to\mathbb{N}$ is a function that maps every symbol from $\Sigma$ to a natural number (its rank).
\end{definition}
If the function \rank is understood we will just write $\Sigma$ instead of $(\Sigma,\rank)$. The set of all elements with a certain rank $k$ in $\Sigma$, denoted by $\Sigma^{(k)}$, is defined by $\Sigma^{(k)}:=rk^{-1}(k)$. In the following we will write $\Sigma=\{P^{(0)},Q^{(3)}\}$ to say that $\Sigma=\{P,Q\}$, $\rank(P)=0$, and $\rank(Q)=3$.
\subsection{First-order logic}
Let $\mathcal{V}=\{x_0,x_1,\dots\}$ be a countable set (of variables), $\mathcal{F}=\{\}$ a ranked set (of function symbols), and $\mathcal{P}=\{\}$ a ranked set (of predicate symbols). The first-order formulas over $(\mathcal{V},\mathcal{F},\mathcal{P})$, are defined as follows: