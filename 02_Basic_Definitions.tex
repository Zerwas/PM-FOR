\section{Basic Definitions}
We will denote the set $\{1,\dots,n\}$ by $\left[n\right]$.
\begin{definition}
A \define{ranked set} is a tuple $(\Sigma,\rank)$, where $\Sigma$ is a countable set and $\rank:\Sigma\to\mathbb{N}$ is a function that maps every symbol from $\Sigma$ to a natural number (its rank).
\end{definition}
If the function \rank{} is understood we will just write $\Sigma$ instead of $(\Sigma,\rank)$. The set of all elements with a certain rank $k$ in $\Sigma$, denoted by $\Sigma^{(k)}$, is defined by $\Sigma^{(k)}:=rk^{-1}(k)$. In the following we will write $\Sigma=\{P^{(0)},Q^{(3)}\}$ to say that $\Sigma=\{P,Q\}$, $\rank(P)=0$, and $\rank(Q)=3$.

First-order logic
\begin{definition}
Let $\mathcal{V}=\{x_0,x_1,\dots\}$ be a countable set (of variables), $\mathcal{F}=\{\}$ a ranked set (of function symbols), and $\mathcal{P}=\{\}$ a ranked set (of predicate symbols). Then the set of \define{terms over $(\mathcal{V},\mathcal{F})$}, denoted by $\mathcal{T}_{(\mathcal{V},\mathcal{F})}$, is the smallest set $\mathcal{T}$ satisfying the following conditions:
\begin{itemize}
\item $\mathcal{V} \subseteq \mathcal{T}$, and
\item for every $k\in\mathbb{N}$ if $f\in\mathcal{F}^{(k)}$ and $t_1,t_2,\dots,t_k\in\mathcal{T}$ then $f(t_1,t_2,\dots,t_k)\in\mathcal{T}$.
\end{itemize}
The set of \define{first-order formulas over $(\mathcal{V},\mathcal{F},\mathcal{P})$}, denoted by $\mathcal{L}_{(\mathcal{V},\mathcal{F},\mathcal{P})}$, is the smallest set $\mathcal{L}$ satisfying the following conditions:
\begin{itemize}
\item for every $k\in\mathbb{N}$ if $P\in\mathcal{P}^{(k)}$ and $t_1,t_2,\dots,t_k\in\mathcal{T}$ then $P(t_1,t_2,\dots,t_k)\in\mathcal{L}$.
\item If $\varphi,\psi\in\mathcal{L}$ then $(\varphi\wedge\psi), (\varphi\vee\psi), (\varphi\to\psi), \neg \varphi\in\mathcal{L}$, and %TODO really include implies?
\item if $x\in\mathcal{V}$ and $\varphi\in\mathcal{L}$ then $\exists x\varphi,\forall x\varphi\in\mathcal{L}$. %TODO dots?
\end{itemize}
\end{definition}
\begin{definition}
The \define{free variables of a formula $\varphi$}, denoted by $\FV(\varphi)$, are defined as follows:
\[\FV(\varphi)=
\begin{cases}
V(t_1)\cup V(t_2)\cup\dots\cup V(t_k) & \text{if $\varphi=P(t_1,t_2,\dots,t_k)$}\\ %TODO V(t)
\FV(\varphi_1)\cup\FV(\varphi_2) & \text{if $\varphi=\varphi_1\circ\varphi_2$, $\circ\in\{\wedge,\vee,\to\}$}\\
\FV(\psi)\setminus\{x\} & \text{if $\varphi=Q x\psi$, $Q\in\{\forall,\exists\}$}
\end{cases}\]
\end{definition}
\begin{definition}
An \define{interpretation $I$ over $(\mathcal{V},\mathcal{F},\mathcal{P})$} is a triple $(\Delta,\cdot^I,\omega)$
\begin{tabular}{llp{0.78\linewidth}}
		where & $\Delta$      & is a set (which we call  domain),                                                                                                       \\
		      & $\cdot^I$ & is a function such that\\
		      & & $f^I:\Delta^k\to\Delta$  is a function for every $k\in\mathbb{N}$, $f\in\mathcal{F}^{(k)}$ and \\
		      & & $P^I\subseteq\Delta^k$ is a relation for every $k\in\mathbb{N}$, $f\in\mathcal{P}^{(k)}$ \\
		      & $\omega$ & is a function from $\mathcal{V}$ to $\Delta$.                       
	\end{tabular}
\end{definition}