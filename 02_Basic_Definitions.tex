\section{Basic Definitions}
We will denote the set $\{1,\dots,n\}$ by $\left[n\right]$.
\subsection{$\lambda$-calculus $\lambda2$}
$\emph{FV}(\Gamma)=\bigcup\{\emph{FV}(t)\mid (x:t)\in \Gamma\}$\\
\begin{definition}
Let $\lambdaTypVar=\{\alpha,\beta,\dots\}$ be a countable set (of type-variables) the \define{set of all $\lambda2$ types over \lambdaTypVar}, denoted by \lambdaTypes{}, is the smallest set T satisfying the following conditions:
\begin{itemize}
\item $\lambdaTypVar\subseteq\text{T}$,
\item if $t_1,t_2\in\text{T}$ then $t_1\to t_2\in\text{T}$, and
\item if $t\in\text{T}$ and $\alpha\in\lambdaTypVar$ then $\forall\alpha.t\in\text{T}$.
\end{itemize}
\end{definition}

\begin{definition}
Let \lambdaTypVar{} be a countable set (of type-variables) and $\lambdaValVar=\{x_1,x_2,\dots\}$ a countable set (of value-variables) the \define{set of all $\lambda2$ terms over \lambdaTypVar{} and \lambdaValVar}, denoted by \lambdaTerms{}, is the smallest set $\Lambda_T$ satisfying the following conditions: %TODO \Lambda_T bad name + lambda2 explicit
\begin{itemize}
\item $\lambdaValVar\subseteq\Lambda_T$,
\item if $e_1,e_2\in\Lambda_T$ then $e_1e_2\in\Lambda_T$,
\item if $x\in\lambdaValVar$, $t\in\lambdaTypes$, and $e\in\Lambda_T$ then $\lambda x:t.e\in\Lambda_T$,
\item if $\alpha\in\lambdaTypVar$ and $e\in\Lambda_T$ then $\Lambda \alpha.e\in\Lambda_T$, and
\item if $e\in\Lambda_T$ and $t\in\lambdaTypes$ then $e\:t\in\Lambda_T$.
\end{itemize}
\end{definition}
$\lambda2$ deduction Rules
\begin{mdframed}
\begingroup
\addtolength{\jot}{0.3cm}
\begin{align*}
&(\text{Axiom}) &&\Gamma,x:t\vdash x:t\\
&(\lambda\text{-Introduction}) &&\frac{\Gamma,x:t_1\vdash e:t_2}{\Gamma\vdash \lambda x.e:t_1\to t_2}\\
&(\lambda\text{-Elimination}) &&\frac{\Gamma\vdash e_1:t_1\to t_2 \hspace{0.4cm}\Gamma\vdash e_2:t_1}{\Gamma\vdash e_1e_2:t_2}\\
&(\forall\text{-Introduction}) &&\frac{\Gamma\vdash e:t}{\Gamma\vdash \Lambda\alpha.e:\forall\alpha.t} &&\alpha\notin\FV(\Gamma) \\
&(\forall\text{-Elimination}) &&\frac{\Gamma\vdash e:\forall\alpha.t }{\Gamma\vdash et':t\left[ \alpha:=t'\right] }
\end{align*}
\endgroup
\end{mdframed}

\subsection{first-order logic}
\begin{definition}
A \define{ranked set} is a tuple $(\Sigma,\rank)$, where $\Sigma$ is a countable set and $\rank:\Sigma\to\mathbb{N}$ is a function that maps every symbol from $\Sigma$ to a natural number (its rank).
\end{definition}
If the function \rank{} is understood we will just write $\Sigma$ instead of $(\Sigma,\rank)$. The set of all elements with a certain rank $k$ in $\Sigma$, denoted by $\Sigma^{(k)}$, is defined by $\Sigma^{(k)}:=rk^{-1}(k)$. In the following we will write $\Sigma=\{P^{(0)},Q^{(3)}\}$ to say that $\Sigma=\{P,Q\}$, $\rank(P)=0$, and $\rank(Q)=3$.

First-order logic
\begin{definition}
Let $\mathcal{V}=\{x_0,x_1,\dots\}$ be a countable set (of variables), $\mathcal{F}=\{\}$ a ranked set (of function symbols), and $\mathcal{P}=\{\}$ a ranked set (of predicate symbols). Then the set of \define{terms over $(\mathcal{V},\mathcal{F})$}, denoted by $\mathcal{T}_{(\mathcal{V},\mathcal{F})}$, is the smallest set $\mathcal{T}$ satisfying the following conditions:
\begin{itemize}
\item $\mathcal{V} \subseteq \mathcal{T}$, and
\item for every $k\in\mathbb{N}$ if $f\in\mathcal{F}^{(k)}$ and $t_1,t_2,\dots,t_k\in\mathcal{T}$ then $f(t_1,t_2,\dots,t_k)\in\mathcal{T}$.
\end{itemize}
The set of \define{first-order formulas over $(\mathcal{V},\mathcal{F},\mathcal{P})$}, denoted by $\mathcal{L}_{(\mathcal{V},\mathcal{F},\mathcal{P})}$, is the smallest set $\mathcal{L}$ satisfying the following conditions:
\begin{itemize}
\item for every $k\in\mathbb{N}$ if $P\in\mathcal{P}^{(k)}$ and $t_1,t_2,\dots,t_k\in\mathcal{T}$ then $P(t_1,t_2,\dots,t_k)\in\mathcal{L}$.
\item If $\varphi,\psi\in\mathcal{L}$ then $(\varphi\wedge\psi), (\varphi\vee\psi), (\varphi\to\psi), \neg \varphi\in\mathcal{L}$, and %TODO really include implies?
\item if $x\in\mathcal{V}$ and $\varphi\in\mathcal{L}$ then $\exists x\varphi,\forall x\varphi\in\mathcal{L}$. %TODO dots?
\end{itemize}
\end{definition}
To define the free variables of a formula we first need to define variables of a term.
\begin{definition}
The \define{variables of a term $t$}, denoted by $\V(t)$, are defined by:
\[\V(t)=
\begin{cases}
\{x\} & \text{if $t=x$}\\
\V(t_1)\cup \V(t_2)\cup\dots\cup \V(t_k) & \text{if $t=f(t_1,t_2,\dots,t_k)$}
\end{cases}\]
\end{definition}

\begin{definition}
The \define{free variables of a formula $\varphi$}, denoted by $\FV(\varphi)$, are defined as follows:
\[\FV(\varphi)=
\begin{cases}
\V(t_1)\cup \V(t_2)\cup\dots\cup \V(t_k) & \text{if $\varphi=P(t_1,t_2,\dots,t_k)$}\\
\FV(\varphi_1)\cup\FV(\varphi_2) & \text{if $\varphi=\varphi_1\circ\varphi_2$, $\circ\in\{\wedge,\vee,\to\}$}\\
\FV(\psi)\setminus\{x\} & \text{if $\varphi=Q x\psi$, $Q\in\{\forall,\exists\}$}
\end{cases}\]
\end{definition}
Now we come to the semantics of first-order formulas.
\begin{definition}
An \define{interpretation $I$ over $(\mathcal{V},\mathcal{F},\mathcal{P})$} is a triple $(\Delta,\cdot^I,\omega)$
\begin{tabular}{llp{0.78\linewidth}}
		where & $\Delta$      & is a set (which we call  domain),                                                                                                       \\
		      & $\cdot^I$ & is a function such that\\
		      & & $f^I:\Delta^k\to\Delta$  is a function for every $k\in\mathbb{N}$, $f\in\mathcal{F}^{(k)}$ and \\
		      & & $P^I\subseteq\Delta^k$ is a relation for every $k\in\mathbb{N}$, $f\in\mathcal{P}^{(k)}$ \\
		      & $\omega$ & is a function from $\mathcal{V}$ to $\Delta$.                       
	\end{tabular}
\end{definition}
%TODO allways write over (...) or make a commnt about it
Let $I=(\Delta,\cdot^I,\omega)$ be an interpretation, $x\in\mathcal{V}$, and $d\in\Delta$ the interpretation $I\left[x\to d\right]$ is defined as $(\Delta,\cdot^I,\omega\left[x\to d\right])$ where
\[(\omega\left[x\to d\right])(y)=
\begin{cases}
d & \text{if $y=x$}\\
\omega(y) & \text{otherwise.}
\end{cases}\]
\begin{definition}
Let $I=(\Delta,\cdot^I,\omega)$ be an interpretation and $t$ a term the \define{interpretation of $t$ under $I$}, denoted by $t^I$, is defined as follows:
\[t^I=
\begin{cases}
\omega(x) & \text{if $t=x$}\\
f^I(t^I_1,\dots,t^I_k) & \text{if $t=f(t_1,\dots,t_k)$}
\end{cases}\]
\end{definition}
\begin{definition}
Let $I=(\Delta,\cdot^I,\omega)$ be an interpretation and $\varphi$ a formula the \define{interpretation of $\varphi$ under $I$}, denoted by $\varphi^I$, is defined recursively as follows:
\[\varphi^I=
\begin{cases}
%TODO quantify k?
\top & \text{if $\varphi=P(t_1,\dots,t_k)$ and $(t^I_1,\dots,t^I_k)\in P^I$}\\
\bot & \text{if $\varphi=P(t_1,\dots,t_k)$ and $(t^I_1,\dots,t^I_k)\notin P^I$}\\
\text{not}~\psi^I & \text{if $\varphi=\neg\psi$}\\
\varphi^I_1~\text{and}~\varphi^I_2 & \text{if $\varphi=(\varphi_1\wedge\varphi_2)$}\\
\varphi^I_1~\text{or}~\varphi^I_2 & \text{if $\varphi=(\varphi_1\vee\varphi_2)$}\\
%TODO how to handle implies?
\text{exists $d\in\Delta$}~\psi^{I\left[x\to d\right]} & \text{if $\varphi=\exists x\psi$}\\
\text{forall $d\in\Delta$}~\psi^{I\left[x\to d\right]} & \text{if $\varphi=\forall x\psi$}\\
\end{cases}\]
The interpretation $I$ is a \define{model} of $\varphi$, denoted by $I\models\varphi$, if $\varphi^I=\top$.
\end{definition}
\begin{definition}
Let $\Gamma$ be a set of first-oder formulas.
\begin{description}
\item We say that an interpretation $I$ is a \define{model} of $\Gamma$ if $I\models\psi$ for every $\psi$ in $\Gamma$.
\item The formula $\varphi$ is a \define{semantic consequence} of $\Gamma$, denoted by $\Gamma\folmodels\varphi$, if every model of $\Gamma$ is also a model of $\varphi$.
\end{description}
\end{definition}