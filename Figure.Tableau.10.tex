\begin{tikzpicture}[grow=up,level distance=0.5cm,
execute at begin node=$, execute at end node=$,
every node/.style={opacity=1},
every child/.style={edge from parent/.style={opacity=0}}]
\def\dist{0.6cm}
\node(e) {\vdashf \falses} [sibling distance=2*\dist+6.4cm] 
	child {node(1) {\vdashf R_1(b,a_1)} [sibling distance=\dist+2.9cm]
		child[xshift=1.7cm] {node(11) {\vdashf P(a_0,a_1)}}
		child[xshift=1.7cm] {node(12) {\vdashf P(a_0,a_1)\to R_1(b,a_1)} [sibling distance=\dist+3.5cm]
			child[xshift=1.7cm] {node(121) {\vdashf D(a_0)}}
			child[xshift=1.7cm] {node(122) {\vdashf D(a_0)\to P(a_0,a_1)\to R_1(b,a_1)} [sibling distance=\dist+4.5cm]
				child[xshift=2cm] {node(1221) {\vdashf R_1(a,a_0)}}
				child[xshift=2cm] {node(1222) {\vdashf R_1(a,a_0)\to D(a_0)\to P(a_0,a_1)\to R_1(b,a_1)} [sibling distance=\dist+5.2cm]
					child[xshift=2cm] {node(12221) {\vdashf S(a,b)}}
					child[xshift=2cm] {node(12222) {\vdashf S(a,b)\to R_1(a,a_0)\to D(a_0)\to P(a_0,a_1)\to R_1(b,a_1)} [sibling distance=\dist+5.9cm]
						child[xshift=2cm] {node(122221) {\vdashf Q(a)}}
						child[xshift=2cm] {node(122222) {\vdashf Q(a)\to S(a,b)\to R_1(a,a_0)\to D(a_0)\to P(a_0,a_1)\to R_1(b,a_1)}
							child {node(1222221) {\vdashf\forall\alpha\beta\gamma\delta(Q(\alpha)\to S(\alpha,\beta)\to R_1(\alpha,\gamma)\to D(\gamma) \to P(\gamma,\delta)\to R_1(\beta,\delta))}}}}}}}}
	child {node(2) {\vdashf R_1(b,a_1)\to\falses}
		child {node(21) {R_1(b,a_1)\vdashf\falses}
			child {node[label={[yshift=-0.3cm]above:\vdots}](211) {}}}};


%tikz did not want to do this in the loop no idea why
%\coordinate(c1222) at (1222);
%\getwidthofnode{\sright}{12221}
%\draw (c1222)+(-\sright/2,+0.25) -- +(\sright/2,+0.25);

%set of all positions in the tree
\def\nodes{,1,2,11,12,21,121,122,1221,1222,12222,12221,122222,122221}
%the yshift does not work on nodes so we create a coordinate for every node
\foreach \x in \nodes{
\ifnodedefined{\x}{
	\coordinate(c\x) at (\x);
}{}
}

%draw lines for each node with 2 children (add parents with 2 children)
\foreach \x in \nodes{
\ifnodedefined{\x1}{
	%calculate with of line
	\getwidthofnode{\sright}{\x1}
	%if there is a right child draw wide line
	\ifnodedefined{\x2}{
		\getwidthofnode{\sleft}{\x2}
		%draw line
		\draw[shorten <=-\sright/2, shorten >=-\sleft/2] ([yshift=-0.25cm]c\x1) -- ([yshift=-0.25cm]c\x2);
	} %else only one child
	{
		\getwidthofnode{\sleft}{\x}
		%get widht corresponding to the widht of the bigger node
		\pgfmathsetlength{\sright}{max(\sright,\sleft)}
		%draw line
		\draw (c\x)+(-\sright/2,+0.25) -- +(\sright/2,+0.25);
	}
}{}
}
\end{tikzpicture}