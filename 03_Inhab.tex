\section{\lambdaInhab{} is undecidable}
Now we can show that the inhabitation problem in $\lambda2$ %explicit?
is undecidable by reducing \PCons{} to \lambdaInhab{}. Given a \SysP-basis $\Gamma$ we construct a \lambdaTwo-basis $\trans{\Gamma}$ such that 
\begin{align*}
\Gamma\PModels\false&&\text{iff}&&\trans{\Gamma}\lambdaModels\false
\end{align*}
where \false{} 

\begin{definition}
We define a function ...

For a \SysP-formula $A$, if $A$ is an atomic formula then
\[
\trans{A}=
\begin{cases}
\false &\text{if $A=\false$}\\
(\alpha\to p_1)\to(\beta\to p_2)\to p &\text{if $A=P(\alpha,\beta)$} %TODO convention with variables greek or latin?
\end{cases}
\]
if $A$ is an universal formula, it follows that there is an $n\in\N$ and atomic formulas $A_1,A_2,\dots,A_n$ such that $A=\forall\vec{\alpha}(A_1\to A_2 \to\dots\to A_n)$, then %TODO quantify varibales
\[\trans{A}=\forall\vec{\alpha}(\trans{A_1}\to \trans{A_2} \to\dots\to\trans{ A_n})\]
if $A$ is an existential formula, it follows that for some $n\in\N^+$ and some atomic formulas $A_1,\dots,A_n$ it holds that $A=\forall\vec{\alpha}(A_1 \to\dots\to A_{n-1}\to \forall\beta((A_n)\to\false)\to\false)$, then %TODO quantify \beta \alpha
\[\trans{A}=\forall\vec{\alpha}(\trans{A_1}\to\dots\to\forall\beta(\trans{ A_n}\to\false)\to\false)\]
For a \SysP-basis $\Gamma$ we define $\trans{\Gamma}$ as $\{(x_A:\trans{A})\mid A\in\Gamma\}$.

\end{definition}

\begin{lemma}
Let $\Gamma$ be a \SysP-basis, $M\in\lambdaTerms$, $P\in\RelP$, and $t_1,t_2\in\lambdaTypes$ such that $\trans{\Gamma}\lambdaModels M:\code{P}{t_1}{t_2}$ holds.
Then $t_1=a$ and $t_2=b$ for some $a,b\in\VarP$ (remember that $\VarP\subseteq\lambdaTypVar$). Furthermore $\Gamma\PModels P(a,b)$ holds.
\end{lemma}
\begin{proof}
Note that all well typed $\lambda2$ terms are strongly normalizing (see %TODO reference
). Hence, $M$ is a well typed in $\lambda2$, we can assume that $M$ is in normal form. %TODO define normal form, and typerhaltungstheorem!

We now proof the lemma by structural induction on the term $M$.
\begin{itemize}
	\item[] \underline{$M=x$} for some $y\in\lambdaValVar$.\\
		It follows that $(x:\code{P}{t_1}{t_2})\in\trans{\Gamma}$.
		Now the definition of $\trans{\Gamma}$ yields that $P(t_1,t_2)\in\Gamma$. Therefore $t_1,t_2\in\VarP$ and $t_1=a$ and $t_2=b$ for some $a,b\in\VarP$. Trivially, $\Gamma\PModels P(a,b)$ holds.
	\item[] \underline{$M=M_1M_2$} for some $M_1,M_2\in\lambdaTerms$.\\
		Since $M$ is in normal form we have that $M_1=xN_1\dots N_k$ for some $x\in\lambdaValVar$, $n\in\N$, and some $N_1,\dots,N_k\in\lambdaTerms\cup\lambdaTypes$. %TODO There is only one kind of %TODO functional neccessary P(a,b)
		%type with target | this is not a target -> $\code{P}{\alpha}{\beta}$ 
		
		Therefore there has to be a universal formula $\forall\vec{\alpha}(A_1\to\dots\to A_n\to P(\alpha,\beta))$ in $\Gamma$, $\vec{\alpha}=(\alpha_1,\dots,\alpha_m)$, $\vec{t}=(t_1,\dots,t_m)$, $A_i=P^i(\alpha_i,\beta_i)$. %TODO t_1,t_2 for 2 different variables!!!
		
		$M=x\vec{t}\vec{N}$, $\trans{\Gamma}\lambdaModels N_i:\widetilde{A_i}$ where $\widetilde{A_i}=\trans{A_i}\left[\vec{\alpha}:=\vec{t}\hspace{1mm}\right]=\code{P^i}{\widetilde{a}_i}{\widetilde{b}_i}$ and $\widetilde{a}_i=\alpha_i\left[\vec{\alpha}:=\vec{t}\hspace{1mm}\right]$ for $i\in\{1,\dots,n\}$
		
		\begin{figure}[H]
			\centering
			\begin{tikzpicture}[grow=up,level distance=0.7cm,
execute at begin node=$, execute at end node=$,
every node/.style={opacity=1},
every child/.style={edge from parent/.style={opacity=0}}]
\def\dist{0.7cm}
\node(e) {\trans{\Gamma}\lambdaModels (x\predTVec N_1\dots N_{n-1})N_n:\code{P}{\predTFst}{\predTSnd}} [sibling distance = 5cm]
	child {node(1) {\trans{\Gamma}\lambdaModels N_n:\code{P^n}{\predTFst[n]}{\predTSnd[n]}}}
	child {node(2) {\trans{\Gamma}\lambdaModels x\predTVec N_1\dots N_{n-1}:\code{P^n}{\predTFst[n]}{\predTSnd[n]}\to\code{P}{\predTFst}{\predTSnd}}
		child {{node[label={[yshift=-0.44cm]above:\vdots}](21) {}}
			child {node(211) {\trans{\Gamma}\lambdaModels N_1:\code{P^1}{\predTFst[1]}{\predTSnd[1]}}}
			child {node(212) {\trans{\Gamma}\lambdaModels x\predTVec:\code{P^1}{\predTFst[1]}{\predTSnd[1]}\to\dots\to\code{P^n}{\predTFst[n]}{\predTSnd[n]}\to\code{P}{\predTFst}{\predTSnd}}
				child {node(2121) {\trans{\Gamma}\lambdaModels x:\forall\predVec(\code{P^1}{\predFst[1]}{\predSnd[1]}\to\dots\to\code{P^n}{\predFst[n]}{\predSnd[n]}\to\code{P}{\predFst}{\predSnd})}}}}};


\def\nodes{,1,2,21,211,212,2121}
\def\identifier{C1:2}
\foreach \x in \nodes{
\ifnodedefined{\x}{
	\coordinate(c\x) at (\x);
	\coordinate(\identifier.\x) at (0,0);
}{}
}
\coordinate(c) at (e);
\def\lvl{3.5mm}

%draw lines for each node with 2 children (add parents with 2 children)
\foreach \x in \nodes{
\ifnodedefined{\identifier.\x1}{
	%calculate with of line
	\getwidthofnode{\sright}{\x1}
	%if there is a right child draw wide line
	\ifnodedefined{\identifier.\x2}{
		\getwidthofnode{\sleft}{\x2}
		%draw line
		\draw[shorten <=-\sright/2, shorten >=-\sleft/2] ([yshift=-\lvl]c\x1) -- ([yshift=-\lvl]c\x2);
	} %else only one child
	{
		\ifnodedefined{\x}
			{\getwidthofnode{\sleft}{\x}}
			{\getwidthofnode{\sleft}{e}}
		%get widht corresponding to the widht of the bigger node
		\pgfmathsetlength{\sright}{max(\sright,\sleft)}
		%draw line
		\draw (c\x)+(-\sright/2,+\lvl) -- +(\sright/2,+\lvl);
	}
}{}
}
\end{tikzpicture}
		\end{figure}
		
		For $i\in\{1,\dots,n\}$ we can now apply the induction hypothesis to $\trans{\Gamma}\lambdaModels N_i:\code{P^i}{\widetilde{a}_i}{\widetilde{b}_i}$ and we get that there exist $a_i,b_i\in\VarP$ such that $\widetilde{a}_i=a_i$, $\widetilde{b}_i=b_i$, and $\Gamma\PModels P^i(a_i,b_i)$.
		
		Since there are no dummy quantifiers it follows that $\vec{t}=\vec{a}$. Either $\alpha$ is in $\vec{\alpha}$ implies $t_1=\alpha\left[\vec{\alpha}:=\vec{t}\hspace{1mm}\right]=a$ or $\alpha$ is free then $\alpha=t_1\in\VarP$ anyway.
		
		\begin{figure}[H]
			\centering
			\begin{tikzpicture}[grow=up,level distance=0.7cm,
execute at begin node=$, execute at end node=$,
every node/.style={opacity=1},
every child/.style={edge from parent/.style={opacity=0}}]
\def\dist{0.7cm}
\node(e) {\Gamma\PModels  P(\predTFst,\predTSnd)} [sibling distance = 5cm]
	child {node(1) {\Gamma\PModels P^ n(\predTFst[n],\predTSnd[n])}}
	child {node(2) {\Gamma\PModels P^ n(\predTFst[n],\predTSnd[n])\to P(a,b)}
		child {{node[label={[yshift=-0.44cm]above:\vdots}](21) {}} [sibling distance = 6cm]
			child[xshift=1.5cm] {node(211) {\Gamma\PModels P^ 1(\predTFst[1],\predTSnd[1])}}
			child[xshift=1.5cm] {node(212) {\Gamma\PModels P^ 1(\predTFst[1],\predTSnd[1])\to\dots\to P^ n(\predTFst[n],\predTSnd[n])\to P(\predTFst,\predTSnd)}
				child {node(2121) {\Gamma\PModels \forall\predVec(P^ 1(\predFst[1],\predSnd[1])\to\dots\to P^n(\predFst[n],\predSnd[n])\to P(\predFst,\predSnd))}}}}};


\def\nodes{,1,2,21,211,212,2121}
\def\identifier{C1:3}
\foreach \x in \nodes{
\ifnodedefined{\x}{
	\coordinate(c\x) at (\x);
	\coordinate(\identifier.\x) at (0,0);
}{}
}
\coordinate(c) at (e);
\def\lvl{3.5mm}

%draw lines for each node with 2 children (add parents with 2 children)
\foreach \x in \nodes{
\ifnodedefined{\identifier.\x1}{
	%calculate with of line
	\getwidthofnode{\sright}{\x1}
	%if there is a right child draw wide line
	\ifnodedefined{\identifier.\x2}{
		\getwidthofnode{\sleft}{\x2}
		%draw line
		\draw[shorten <=-\sright/2, shorten >=-\sleft/2] ([yshift=-\lvl]c\x1) -- ([yshift=-\lvl]c\x2);
	} %else only one child
	{
		\ifnodedefined{\x}
			{\getwidthofnode{\sleft}{\x}}
			{\getwidthofnode{\sleft}{e}}
		%get widht corresponding to the widht of the bigger node
		\pgfmathsetlength{\sright}{max(\sright,\sleft)}
		%draw line
		\draw (c\x)+(-\sright/2,+\lvl) -- +(\sright/2,+\lvl);
	}
}{}
}
\end{tikzpicture}
		\end{figure}
		
	\item[] \underline{$M=\lambda x:t'.M'$} for some $M'\in\lambdaTerms$, some $x\in\lambdaValVar\setminus\dom(\Gamma)$, and some $t'\in\lambdaTypes$.\\
		It follows that $t'=t_1\to p_1$ and $\trans{\Gamma},(x:t_1\to p_1)\lambdaModels M':(t_2\to p_2)\to p$
		%TODO possible M'=yx; y:(t1->p1)->(t2->p2)->p; eta reduction?
		%or M'=\y:t2->p2.M''; G,x,y |- M'':p
		
		\begin{figure}[H]
			\centering
			\begin{tikzpicture}[grow=up,level distance=0.7cm,
execute at begin node=$, execute at end node=$,
every node/.style={opacity=1},
every child/.style={edge from parent/.style={opacity=0}}]
\def\dist{0.7cm}
\node(e) {\Gamma\PModels  P(\predTFst,\predTSnd)} [sibling distance = 5cm]
	child {node(1) {\Gamma\PModels P^ n(\predTFst[n],\predTSnd[n])}}
	child {node(2) {\Gamma\PModels P^ n(\predTFst[n],\predTSnd[n])\to P(a,b)}
		child {{node[label={[yshift=-0.44cm]above:\vdots}](21) {}} [sibling distance = 6cm]
			child[xshift=1.5cm] {node(211) {\Gamma\PModels P^ 1(\predTFst[1],\predTSnd[1])}}
			child[xshift=1.5cm] {node(212) {\Gamma\PModels P^ 1(\predTFst[1],\predTSnd[1])\to\dots\to P^ n(\predTFst[n],\predTSnd[n])\to P(\predTFst,\predTSnd)}
				child {node(2121) {\Gamma\PModels \forall\predVec(P^ 1(\predFst[1],\predSnd[1])\to\dots\to P^n(\predFst[n],\predSnd[n])\to P(\predFst,\predSnd))}}}}};


\def\nodes{,1,2,21,211,212,2121}
\def\identifier{C1:3}
\foreach \x in \nodes{
\ifnodedefined{\x}{
	\coordinate(c\x) at (\x);
	\coordinate(\identifier.\x) at (0,0);
}{}
}
\coordinate(c) at (e);
\def\lvl{3.5mm}

%draw lines for each node with 2 children (add parents with 2 children)
\foreach \x in \nodes{
\ifnodedefined{\identifier.\x1}{
	%calculate with of line
	\getwidthofnode{\sright}{\x1}
	%if there is a right child draw wide line
	\ifnodedefined{\identifier.\x2}{
		\getwidthofnode{\sleft}{\x2}
		%draw line
		\draw[shorten <=-\sright/2, shorten >=-\sleft/2] ([yshift=-\lvl]c\x1) -- ([yshift=-\lvl]c\x2);
	} %else only one child
	{
		\ifnodedefined{\x}
			{\getwidthofnode{\sleft}{\x}}
			{\getwidthofnode{\sleft}{e}}
		%get widht corresponding to the widht of the bigger node
		\pgfmathsetlength{\sright}{max(\sright,\sleft)}
		%draw line
		\draw (c\x)+(-\sright/2,+\lvl) -- +(\sright/2,+\lvl);
	}
}{}
}
\end{tikzpicture}
		\end{figure}
	\item[] \underline{$M=\Lambda\gamma.M'$} for some $M'\in\lambdaTerms$ and some $\gamma\in\lambdaTypVar\setminus\FV(\Gamma)$.\\
		It follows that $t=\forall\gamma.t'$ for some $t'\in\lambdaTypes$.
		But this can not be since $\code{P}{t_1}{t_2}=(t_1\to p_1)\to(t_2\to p_2)\to p$. Therefore $M$ is not of the form $\Lambda\gamma.M'$ and this case is impossible.

	\item[] \underline{$M=M'\,t'$} for some $M'\in\lambdaTerms$ and some $t'\in\lambdaTypes$.\\
		Since $M$ is in normal form we have that either $M'=x$ for some $x\in\lambdaValVar$ and $(x:\forall\beta.t'')\in\trans{\Gamma}$ for some $\beta\in\lambdaTypVar$ and some $t''\in\lambdaTerms$ such that $t=t''\left[\beta:=t'\right]$ or $M'=xM_1\dots M_n$ for some $x\in\lambdaValVar$, $n\in\N$, and some $M_1,\dots,M_n\in\lambdaTerms\cup\lambdaTypes$.
		
		If $n>1$ impossible
		
		If $n=1$ $M_1=t''$
		
		Obviously, this is impossible if $t'\notin\VarP$. But even if $t'\in\VarP$ we have a contradiction because there are no dummy quantifiers and so there is no \SysP-formula $A$ such that $\trans{A}=t''$. %TODO since the scnd condition cannot be fullfilled rephrase
		It follows that this case is impossible.
\end{itemize}
\end{proof}

\begin{lemma}
Let $\Gamma$ be a \SysP-basis, $M\in\lambdaTerms$ such that $\trans{\Gamma}\lambdaModels M:\false$ holds. Then $\Gamma\PModels\false$ holds.
\end{lemma}
\begin{proof}
By structural induction on the term $M$.
Again we can assume that $M$ is in normal form.
\begin{itemize}
	\item[] \underline{$M=x$} for some $y\in\lambdaValVar$.\\
		It follows that $(x:\false)\in\trans{\Gamma}$. Now the definition of $\trans{\Gamma}$ yields that $\false\in\Gamma$. Therefore $\Gamma\PModels \false$ holds.
	\item[] \underline{$M=M_1M_2$} for some $M_1,M_2\in\lambdaTerms$.\\
		
		\begin{figure}[H]
			\centering
			\begin{tikzpicture}[grow=up,level distance=0.7cm,
execute at begin node=$, execute at end node=$,
every node/.style={opacity=1},
every child/.style={edge from parent/.style={opacity=0}}]
\def\dist{0.7cm}
\node(e) {\trans{\Gamma}\lambdaModels (x\predTVec N_1\dots N_{n-1})N_n:\code{P}{\predTFst}{\predTSnd}} [sibling distance = 5cm]
	child {node(1) {\trans{\Gamma}\lambdaModels N_n:\code{P^n}{\predTFst[n]}{\predTSnd[n]}}}
	child {node(2) {\trans{\Gamma}\lambdaModels x\predTVec N_1\dots N_{n-1}:\code{P^n}{\predTFst[n]}{\predTSnd[n]}\to\code{P}{\predTFst}{\predTSnd}}
		child {{node[label={[yshift=-0.44cm]above:\vdots}](21) {}}
			child {node(211) {\trans{\Gamma}\lambdaModels N_1:\code{P^1}{\predTFst[1]}{\predTSnd[1]}}}
			child {node(212) {\trans{\Gamma}\lambdaModels x\predTVec:\code{P^1}{\predTFst[1]}{\predTSnd[1]}\to\dots\to\code{P^n}{\predTFst[n]}{\predTSnd[n]}\to\code{P}{\predTFst}{\predTSnd}}
				child {node(2121) {\trans{\Gamma}\lambdaModels x:\forall\predVec(\code{P^1}{\predFst[1]}{\predSnd[1]}\to\dots\to\code{P^n}{\predFst[n]}{\predSnd[n]}\to\code{P}{\predFst}{\predSnd})}}}}};


\def\nodes{,1,2,21,211,212,2121}
\def\identifier{C1:2}
\foreach \x in \nodes{
\ifnodedefined{\x}{
	\coordinate(c\x) at (\x);
	\coordinate(\identifier.\x) at (0,0);
}{}
}
\coordinate(c) at (e);
\def\lvl{3.5mm}

%draw lines for each node with 2 children (add parents with 2 children)
\foreach \x in \nodes{
\ifnodedefined{\identifier.\x1}{
	%calculate with of line
	\getwidthofnode{\sright}{\x1}
	%if there is a right child draw wide line
	\ifnodedefined{\identifier.\x2}{
		\getwidthofnode{\sleft}{\x2}
		%draw line
		\draw[shorten <=-\sright/2, shorten >=-\sleft/2] ([yshift=-\lvl]c\x1) -- ([yshift=-\lvl]c\x2);
	} %else only one child
	{
		\ifnodedefined{\x}
			{\getwidthofnode{\sleft}{\x}}
			{\getwidthofnode{\sleft}{e}}
		%get widht corresponding to the widht of the bigger node
		\pgfmathsetlength{\sright}{max(\sright,\sleft)}
		%draw line
		\draw (c\x)+(-\sright/2,+\lvl) -- +(\sright/2,+\lvl);
	}
}{}
}
\end{tikzpicture}
		\end{figure}
	\item[] \underline{$M=\lambda x:t_1.M'$} for some $M'\in\lambdaTerms$, some $x\in\lambdaValVar\setminus\dom(\Gamma)$, and some $t_1\in\lambdaTypes$.\\
		It follows that $t=t_1\to t_2$ for some $t_2\in\lambdaTypes$ which contradicts $t=\false$. So this case is impossible.

	\item[] \underline{$M=\Lambda\gamma.M'$} for some $M'\in\lambdaTerms$ and some $\gamma\in\lambdaTypVar\setminus\FV(\Gamma)$.\\
		It follows that $t=\forall\gamma.t'$ for some $t'\in\lambdaTypes$. Again the fact that $t=\false$ leads to a contradiction and makes this case impossible.
		

	\item[] \underline{$M=M'\,t'$} for some $M'\in\lambdaTerms$ and some $t'\in\lambdaTypes$.\\
		
\end{itemize}
\end{proof}