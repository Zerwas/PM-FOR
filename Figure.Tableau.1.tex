\begin{tikzpicture}[grow=up,level distance=0.5cm,
execute at begin node=$, execute at end node=$,
every node/.style={opacity=1},
every child/.style={edge from parent/.style={opacity=0}}]
\node(e) {\PModelsf \falses} [sibling distance=8cm] 
	child [sibling distance=4cm] {node(1) {\PModelsf\forall\beta(S(a,\beta)\to\falses)\to\falses} 
		child {node(11) {\PModelsf\forall\alpha(\forall\beta(S(\alpha,\beta)\to\falses)\to\falses)}}}
	child {node(2) {\PModelsf(\forall\beta(S(a,\beta)\to\falses)\to\falses)\to\falses}
		child {node(21) {\forall\beta(S(a,\beta)\to\falses)\to\falses\PModelsf\falses}
			child {node(211) {S(a,b)\PModelsf\falses}
				child {node[label={[yshift=-0.3cm]above:\vdots}](2111) {}}}}};

%set of all positions in the tree
\def\nodes{,1,2,11,21,211,2111}
%the yshift does not work on nodes so we create a coordinate for every node
\foreach \x in \nodes{
\ifnodedefined{\x}{
	\coordinate(c\x) at (\x);
	\coordinate(1.\x) at (0,0); %we need this because defined nodes are not fergotten between tikz pictures
}{}
}

%draw lines for each node with 2 children (add parents with 2 children)
\foreach \x in \nodes{
\ifnodedefined{1.\x1}{
	%calculate with of line
	\getwidthofnode{\sright}{\x1}
	%if there is a right child draw wide line
	\ifnodedefined{1.\x2}{
		\getwidthofnode{\sleft}{\x2}
		%draw line
		\draw[shorten <=-\sright/2, shorten >=-\sleft/2] ([yshift=-0.25cm]c\x1) -- ([yshift=-0.25cm]c\x2);
	} %else only one child
	{
		\getwidthofnode{\sleft}{\x}
		%get widht corresponding to the widht of the bigger node
		\pgfmathsetlength{\sright}{max(\sright,\sleft)}
		%draw line
		\draw (c\x)+(-\sright/2,+0.25) -- +(\sright/2,+0.25);
	}
}{}
}

\end{tikzpicture}